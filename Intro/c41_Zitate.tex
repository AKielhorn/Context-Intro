% !TEX TS-program = ConTeXt (LMTX2022)
% !TEX encoding = UTF-8 Unicode

% Copyright 2018 Axel Kielhorn
% Lizenz: CC-BY-SA 4.0 Unported http://creativecommons.org/licenses/by-sa/4.0/deed.de

\startcomponent *
\product prd_intro
\project project_intro


\startsection[
  reference=zitate,
  title={Zitate},
  ]
\index{Zitate}\index{Anführungszeichen}
Längere Zitate werden in einer \type{blockquote} Umgebung gesetzt.

\starttyping
\startblockquote
Hier ist eine Zitat in etwas kleinerer Schrift mit breiteren Rändern.
Es muss noch etwas länger werden, damit man die Ränder besser sieht.
\stopblockquote
\stoptyping

\startblockquote
Hier ist eine Zitat in etwas kleinerer Schrift mit breiteren Rändern.
Es muss noch etwas länger werden, damit man die Ränder besser sieht.
\stopblockquote


Für kurze Zitate, die innerhalb eines Absatzes gibt es den \tex{quotation} Befehl. 
Damit kann man wörtliche Rede \quotation{So wie diese hier.} wiedergeben.
Die Anführungsstriche werden richtig gesetzt. "Hier zum Unterschied falsche Anführungszeichen."

Innerhalb der \tex{quotation} kann man mit dem \tex{quote} Befehl eine weitere Ebene schachteln.
\quotation{Äußere Rede \quote{Innerer Monolog} und hier wieder außen.}

Die Anführungszeichen werden abhängig von der Sprache angepasst.

{\language[fr]
In Französisch sieht das dann so aus:
Innerhalb der \tex{quotation} kann man mit dem \tex{quote} Befehl eine weitere Ebene schachteln.
\quotation{Äußere Rede \quote{Innerer Monolog} und hier wieder außen.}
}

In der Environment-Datei kann man das Aussehen umdefinieren.

\starttyping
\setupdelimitedtext[quotation:1][left=»,right=«]
\setupdelimitedtext[quotation:2][left=›,right=‹]
\stoptyping

\stopsection

\stopcomponent

