% !TEX TS-program = ConTeXt (LMTX2022)
% !TEX encoding = UTF-8 Unicode

% Copyright 2018 Axel Kielhorn
% Lizenz: CC-BY-SA 4.0 Unported http://creativecommons.org/licenses/by-sa/4.0/deed.de

\startcomponent *
\product prd_intro
\project project_intro

\startsection[reference=gliederung,
  title={Gliederung},
  ]
  
  \index{Section}
  \index{Part}
  
  Es gibt zwei Möglichkeiten Gliederungen zu definieren.
  ConTeXt versteht die LaTeX-Befehle \type{\chapter{}}, \type{\section{}} usw.
  
  Für einige Funktionen (tagged-PDF, ePUB Export und XML Export) muss die Gliederung als Umgebung definiert werden.
  
\starttyping
  \startsection[
    reference=gliederung,
    title={Gliederung},
    ]
    Lorem ipsum dolor
  \stopsection
\stoptyping


Ein Dokument hat dann z.\,B. folgenden Aufbau:

\starttyping
\startext
  \startpart[title={Part}]
    \startchapter[title={Chapter}]
      \startsection[title={Section}]
        \startsubsection[title={Subsection}]
          \startsubsubsection[title={Subsubsection}]
            \startsubsubsubsection[title={Subsubsubsection}]
              \startsubsubsubsubsection[title={Subsubsubsubsection}]

              \stopsubsubsubsubsection
            \stopsubsubsubsection
          \stopsubsubsection
        \stopsubsection
      \stopsection
    \stopchapter
  \stoppart
\stoptext
\stoptyping


  \startsubsection[
    reference=subsection,
    title={subsection},
    ]

    Dies ist der Test für eine \type{subsection}.

    \startsubsubsection[
      reference=subsubsection,
      title={subsubsection},
      ]

      Dies ist der Test für eine \type{subsubsection}.

      \startsubsubsubsection[
        reference=subsubsubsection,
        title={subsubsubsection},
        ]

        Dies ist der Test für einen \type{subsubsubsection}. 
        Ab dieser Ebene werden die Abschnitte nicht mehr nummeriert.
        Das läßt sich jedoch leicht mit dem Befehl \type{\setuphead[subsubsubsection][number=yes]} ändern.
        Die entsprechenden Einstellungen finden sich in der Environment-Datei.


        \startsubsubsubsubsection[
          reference= subsubsubsubsection,
          title={subsubsubsubsection},
          ]

          Dies ist der Test für einen \type{subsubsubsubsection}.
        \stopsubsubsubsubsection
      \stopsubsubsubsection
    \stopsubsubsection
  \stopsubsection
  \startsubsection[
    reference=subsection,
    title={Eine ganz lange Überschrift\\ mit manuellen Umbruch},
    list={Eine ganz lange Überschrift ohne manuellen Umbruch},
    ]
  \stopsubsection

Es können unterschiedliche Texte für die Überschrift und den Eintrag im Inhaltsverzeichnis benutzt werden.
So können z.\,B. Formeln oder Sonderzeichen in der Überschrift stehen, ein Beispiel dafür findet sich im Abschnitt \in{Kapitel}[bronx].

\starttyping
  \startsubsection[
    reference=subsection,
    title={Eine ganz lange Überschrift\\ mit manuellen Umbruch},
    list={Eine ganz lange Überschrift ohne manuellen Umbruch},
    ]
\stoptyping


\stopsection

\stopcomponent
