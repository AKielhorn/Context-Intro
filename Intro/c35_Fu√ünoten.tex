% !TEX TS-program = ConTeXt (LuaTeX 1.0.9)
% !TEX encoding = UTF-8 Unicode

% Copyright 2018 Axel Kielhorn
% Lizenz: CC-BY-SA 4.0 Unported http://creativecommons.org/licenses/by-sa/4.0/deed.de

\startcomponent *
\product prd_intro
\project project_intro

\startsection[
  reference=fußnoten,
  title={Fußnoten}
  ]
\index{Fußnoten}

Etwas Fülltext zum Anfang, damit die Fußnoten in der Mitte des Textes
stehen. Ganz „schlimm“ sind lange Fußnoten\footnote{Das heißt, Fußnoten,
  die nicht nur über ein oder zwei Zeile gehen, sondern die Hälfte einer
  Seite einnehmen. Das gibt es in vielen Bereichen, in denen entweder
  jedes Zitat mittels Fußnote belegt wird oder ausführliche Anmerkungen
  in den Fußnoten stehen. Ein ehemaliger Professor von mir hat es in
  einem zweihundertseitigen Skript immerhin auf 200 Fußnoten gebracht.
  Das ist ggf. nicht viel, ich erzähle es aber hier, um die Fußnote
  weiter aufzublähen ohne Blindtext zu benutzen.}.
Oder\footnote{Zum Thema Blindtext kann ich auch noch etwas sagen. Es
  gibt verschiedene Pakete, die Blindtext erzeugen. Darunter das Paket
  \type{blindtext} und das Paket \type{lipsum}. So, und endlich habe ich
  den Umbruch der Fußnote erreicht \ldots{} Argh, nein doch nicht. Je
  nachdem, wie der Text auf der Seite und umbricht, brauchen wir also
  doch noch mehr Text, sodass die Fußzeile weiter wächst und wächst,
  sodass dieser Text hier erst auf der nächsten Seite zu finden ist. So,
  jetzt reicht's! Es gibt also doch Blindtext!} Blindtext\footnote{Die
  fehlende Fußnote 3, die im PDF einfach verschluckt wird, stört aber
  schon etwas. Ich frage mich, wo diese im EPUB auftauchen wird.}
in\footnote{Lorem ipsum dolor sit amet, consectetur adipisici elit, sed
  eiusmod tempor incidunt ut labore et dolore magna aliqua. Ut enim ad
  minim veniam, quis nostrud exercitation ullamco laboris nisi ut
  aliquid ex ea commodi consequat. Quis aute iure reprehenderit in
  voluptate velit esse cillum dolore eu fugiat nulla pariatur.} der
Fußnote\footnote{Excepteur sint obcaecat cupiditat non proident, sunt in
  culpa qui officia deserunt mollit anim id est laborum. Duis autem vel
  eum iriure dolor in hendrerit in vulputate velit esse molestie
  consequat, vel illum dolore eu feugiat nulla facilisis at vero eros et
  accumsan et iusto odio dignissim qui blandit praesent luptatum zzril
  delenit augue duis dolore te feugait nulla facilisi. Lorem ipsum dolor
  sit amet, consectetuer adipiscing elit, sed diam nonummy nibh euismod
  tincidunt ut laoreet dolore magna aliquam erat volutpat. Lorem ipsum
  dolor sit amet, consectetur adipisici elit, sed eiusmod tempor
  incidunt ut labore et dolore magna aliqua. Ut enim ad minim veniam.}.
Und damit hinter dem Text der Fußnoten noch etwas kommt, muss ich mir
hier noch mehr Unsinn einfallen lassen, sodass ich mindestens drei oder
mehr Zeilen zusammenbekomme, die dafür sorgen, dass das alles korrekt
dargestellt wird und aussieht und so weiter und so fort.

\stopsection

\stopcomponent

