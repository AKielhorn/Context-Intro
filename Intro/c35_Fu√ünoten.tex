% !TEX TS-program = ConTeXt (LuaTeX 1.0.9)
% !TEX encoding = UTF-8 Unicode

% Copyright 2018 Axel Kielhorn
% Lizenz: CC-BY-SA 4.0 Unported http://creativecommons.org/licenses/by-sa/4.0/deed.de

\startcomponent *
\product prd_intro
\project project_intro

\definenote[authornote][footnote]      
  \setupnotation[authornote]      [headstyle=bold]
  \setupnote    [authornote]      [textstyle=bold]
  
\definenote[translaternote][rule=off]  
  \setupnotation[translaternote]  [numberconversion=a,   headstyle=bold, headcolor=green, style=bolditalic] %Note Text
  \setupnote[translaternote][textstyle=\bf, textcolor=green] % Note Symbol

\definenote[translaterpnote][rule=off] 
  \setupnotation[translaterpnote] [numberconversion=set 2,headstyle=bold, headcolor=green, style=bolditalic, way=bypage]
  \setupnote[translaterpnote][textstyle=\bf, textcolor=green] % Note Symbol

\definenote[commentnote][endnote]      
  \setupnotation[commentnote]     [color=red,left={(},right={)}]
  \setupnote [commentnote] [textcommand=\mycommentcommand]
    \define[1]\mycommentcommand{\high{\tfxx(#1)}}

%\setupnote[footnote][%
%rule=off,
%frame=on,
%framecolor=green,
%rulecolor=green,
%rulethickness=1pt,
%textstyle=\bf,
%textcolor=green,
%]
%\setupnotation[footnote][%
%numberconversion=a,
%headstyle=bold,
%headcolor=green,
%style=bolditalic,
%color=green,
%]

\startsection[
  reference=fußnoten,
  title={Fußnoten}
  ]
\index{Fußnoten}

Fußnoten werden mit den \type{\footnote{}}\index{footnote} Befehl erzeugt.
Das Aussehen des Fußnotensymbols im Text wird mit dem Befehl \type{\setupnote} definiert.
Die Fußnote selbst wird mit dem Befehl \type{\setupnotation} definiert.
Das Aussehen der Fußnoten lässt sich leicht ändern.

\starttyping
\setupnote[footnote][%
rule=off, % Rule oder Frame
frame=on, % Beides gleichzeitig sieht komisch aus
framecolor=green,
rulecolor=green,
rulethickness=1pt, % für rule und frame
textstyle=\bf,
textcolor=green
]

\setupnotation[footnote][%
numberconversion=a,
headstyle=bold,
headcolor=green,
style=bolditalic,
color=green
]
\stoptyping

Außerdem ist es möglich eigene Fußnotenapparate zu definieren.
So kann zwischen Originalfußnoten und Fußnoten des Übersetzers unterschieden werden.

\starttyping
\definenote[authornote][footnote] % Gleicher Zähler wie footnote     
  \setupnotation[authornote]      [headstyle=bold] % Symbol fett
  \setupnote    [authornote]      [textstyle=bold]
  
\definenote[translaternote][rule=off] % Eigener Zähler
  \setupnotation[translaternote] [
    numberconversion=a,   
    headstyle=bold, headcolor=green,
    style=bolditalic] %Note Text
  \setupnote[translaternote] [
    textstyle=\bf, textcolor=green] % Note Symbol

\definenote[translaterpnote][rule=off] 
  \setupnotation[translaterpnote] [
    numberconversion=set 2,
    headstyle=bold, headcolor=green,
    style=bolditalic,
    way=bypage] % seitenweise numerieren
  \setupnote[translaterpnote] [
    textstyle=\bf, textcolor=green] % Note Symbol
\stoptyping

Außer den Fußnoten gibt es auch noch Endnoten.\index{Endnoten}
Diese werden mit Klammern markiert.

\starttyping
\definenote[commentnote][endnote]      
  \setupnotation[commentnote]     [color=red,left={(},right={)}]
  \setupnote [commentnote] [textcommand=\mycommentcommand]
    \define[1]\mycommentcommand{\high{\tfxx(#1)}}
\stoptyping

Endnoten werden nicht automatisch ausgegeben.
An der gewünschten Stelle muss der Befehl

\starttyping
\placenotes[commentnote]
\stoptyping

aufgerufen werden.

\input ward\authornote{Eine normale Fußnote in Originaltext. Mit etwas Text, damit die Seite voll wird. \input ward}

\input ward\translaternote{Eine Fußnote des Übersetzers.
Sie benutzt einen anderen Zähler als die \type{\authornode} und wird mit Buchstaben statt Zahlen markiert.}
\commentnote{Mehr Informationen gibt es in den Endnoten am Ende des Kapitels.}

\input ward\translaterpnote{Alternativ kann man die Übersetzernoten auch seitenweise nummerieren.
Zur Unterscheidung von normalen Fußnoten hier mit Zeichen statt Zahlen.}

\input ward\footnote{Eine normale Fußnote. 
Sie benutzt den gleichen Zähler wie \type{\authornote}, aber das Symbol ist nicht fett.
Das führt zur Verwirrung und sollte vermieden werden.}

\input ward\commentnote{Noch eine Endnote, damit es sich lohnt.} 

\input ward\translaterpnote{Die zweite Fußnote des Übersetzers auf der Seite.} 

\startsubsection[title={Endnoten}]

\placenotes[commentnote]

\stopsubsection

\stopsection

\stopcomponent

