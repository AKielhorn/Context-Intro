% !TEX TS-program = ConTeXt (LuaTeX 1.0.9)
% !TEX encoding = UTF-8 Unicode

% Copyright 2018 Axel Kielhorn
% Lizenz: CC-BY-SA 4.0 Unported http://creativecommons.org/licenses/by-sa/4.0/deed.de

\startcomponent *
\product prd_intro
\project project_intro

\startsection[
  reference=auflistungen,
  title={Auflistungen}
  ]
\index{Auflistung}

\startparagraph
Hier ist eine einfach Auflistung.
Die Liste wird dreifach verschachtelt.
\stopparagraph

\starttyping
\startitemize
\item
   Listenpunkt 1 
\item
   Listenpunkt 2 

  \startitemize
  \item
     Listenpunkt 2.1 
  \item
     Listenpunkt 2.2 
  \stopitemize
\item
   Listenpunkt 3 
\item
   Listenpunkt 4 

  \startitemize
  \item
     Listenpunkt 4.1 
  \item
     Listenpunkt 4.2 

    \startitemize
    \item
       Listenpunkt 4.2.1 
    \item
       Listenpunkt 4.2.2 
    \stopitemize
  \stopitemize
\stopitemize
\stoptyping

\index{itemize}
\startitemize
\item
   Listenpunkt 1 
\item
   Listenpunkt 2 

  \startitemize
  \item
     Listenpunkt 2.1 
  \item
     Listenpunkt 2.2 
  \stopitemize
\item
   Listenpunkt 3 
\item
   Listenpunkt 4 

  \startitemize
  \item
     Listenpunkt 4.1 
  \item
     Listenpunkt 4.2 

    \startitemize
    \item
       Listenpunkt 4.2.1 
    \item
       Listenpunkt 4.2.2 
    \stopitemize
  \stopitemize
\stopitemize

\startparagraph
Es folgt noch eine Liste, bei der der Listenpunkt mit $\diamond$ überschrieben wurde.
 Dazu definiert man ein neues Symbol und verwendet es in der \type{itemize} Umgebung.
\stopparagraph

\starttyping
{
\definesymbol[itemneu][{$\diamond$}]
\startitemize[symbol=itemneu]
\item
  Listenpunkt 1
\item
  Listenpunkt 2
\stopitemize
}
\stoptyping

{
\definesymbol[itemneu][{$\diamond$}]
\startitemize[symbol=itemneu]
\item
  Listenpunkt 1
\item
  Listenpunkt 2
\stopitemize
}
\stopsection

\stopcomponent

