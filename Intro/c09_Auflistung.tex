% !TEX TS-program = ConTeXt (LuaTeX 1.0.9)
% !TEX encoding = UTF-8 Unicode

% Copyright 2018 Axel Kielhorn
% Lizenz: CC-BY-SA 4.0 Unported http://creativecommons.org/licenses/by-sa/4.0/deed.de

\startcomponent *
\product prd_intro
\project project_intro

\startsection[
  reference=auflistungen,
  title={Auflistungen},
  ]
\index{Auflistung}

\startparagraph
Hier ist eine einfach Auflistung.
Die Liste wird dreifach verschachtelt.
\stopparagraph

\starttyping
\startitemize
\item
   Listenpunkt 1 
\item
   Listenpunkt 2 

  \startitemize
  \item
     Listenpunkt 2.1 
  \item
     Listenpunkt 2.2 
  \stopitemize
\item
   Listenpunkt 3 
\item
   Listenpunkt 4 

  \startitemize
  \item
     Listenpunkt 4.1 
  \item
     Listenpunkt 4.2 

    \startitemize
    \item
       Listenpunkt 4.2.1 
    \item
       Listenpunkt 4.2.2 
    \stopitemize
  \stopitemize
\stopitemize
\stoptyping

\index{itemize}
\startitemize
\item
   Listenpunkt 1 
\item
   Listenpunkt 2 

  \startitemize
  \item
     Listenpunkt 2.1 
  \item
     Listenpunkt 2.2 
  \stopitemize
\item
   Listenpunkt 3 
\item
   Listenpunkt 4 

  \startitemize
  \item
     Listenpunkt 4.1 
  \item
     Listenpunkt 4.2 

    \startitemize
    \item
       Listenpunkt 4.2.1 
    \item
       Listenpunkt 4.2.2 
    \stopitemize
  \stopitemize
\stopitemize

ConTeXt kennt bereits einige Symbole, die als Alternative für das Listensymbol benutzt werden können.

\starttabulate[|T|ch{\symbol}|]
\NC \rmbf Name    \NC \rmbf Symbol    \NC\NR
\TB
\NC bullet        \HC {bullet}        \NC\NR
\NC dash          \HC {dash}          \NC\NR
\NC star          \HC {star}          \NC\NR
\NC triangle      \HC {triangle}      \NC\NR
\NC circle        \HC {circle}        \NC\NR
\NC square        \HC {square}        \NC\NR
\NC diamond       \HC {diamond}       \NC\NR
\NC checkmark     \HC {checkmark}     \NC\NR
\NC asterisk      \HC {asterisk}      \NC\NR
\TB
\NC blacktriangle \HC {blacktriangle} \NC\NR
\NC blacksquare   \HC {blacksquare}   \NC\NR
%\NC blackdiamond  \HC {blackdiamond}  \NC\NR
\TB
\NC 1             \HC {1}             \NC\NR
\NC 2             \HC {2}             \NC\NR
\NC 3             \HC {3}             \NC\NR
\NC 4             \HC {4}             \NC\NR
\NC 5             \HC {5}             \NC\NR
\NC 6             \HC {6}             \NC\NR
\NC 7             \HC {7}             \NC\NR
\NC 8             \HC {8}             \NC\NR
\NC 9             \HC {9}             \NC\NR
\stoptabulate

Die Symbole 1 -- 9 werden für die entsprechenden Auflistungsebenen genutzt.

\starttyping
\startitemize[diamond]
\item
  Listenpunkt 1
\item
  Listenpunkt 2
\stopitemize
\stoptyping

\startitemize[diamond]
\item
  Listenpunkt 1
\item
  Listenpunkt 2
\stopitemize

Reichen diese Symbole nicht aus, so lassen sich eigene definieren.
Hier z.\,B. ein beliebiges Unicodezeichen.

\starttyping
\definesymbol[itemneu][⚭]
\startitemize[itemneu]
\item
  Listenpunkt 1
\item
  Listenpunkt 2
\stopitemize
\stoptyping

\definesymbol[itemneu][⚭]
\startitemize[itemneu]
\item
  Listenpunkt 1
\item
  Listenpunkt 2
\stopitemize

\stopsection

\stopcomponent

