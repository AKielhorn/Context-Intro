% !TEX TS-program = ConTeXt (LuaTeX 1.0.9)
% !TEX encoding = UTF-8 Unicode

% Copyright 2018 Axel Kielhorn
% Lizenz: CC-BY-SA 4.0 Unported http://creativecommons.org/licenses/by-sa/4.0/deed.de

\startcomponent *
\product prd_intro
\project project_intro

\startsection[
  reference=Grafiken,
  title={Grafiken},
  ]
\index{Bilder}\index{Grafik}
Bilder werden mit dem Befehl \type{\externalfigure} eingebunden. 
Dabei kann das Bild skaliert werden.

\starttyping
\externalfigure[cow.pdf]
\externalfigure[cow.pdf][width=3cm]
\externalfigure[cow.pdf][height=2cm]
\externalfigure[cow.pdf][scale=500] % 0,5 * 1000
\externalfigure[cow.pdf][maxwidth=0.4\textwidth]
\stoptyping

\externalfigure[cow.pdf]
\externalfigure[cow.pdf][width=1cm]
\externalfigure[cow.pdf][height=1cm]
\externalfigure[cow.pdf][scale=500]
\externalfigure[cow.pdf][maxwidth=0.4\textwidth]

Damit die Kuh nicht die Seite verlässt, 
empfiehlt es sich in der Environment-Datei folgende Einstellung vorzunehmen:

\starttyping
\setupexternalfigures
    [maxwidth=\textwidth,
     maxheight=0.8\textheight]
\stoptyping

Diese Kuh wird bei ConTeXt als Beispielgrafik mitgeliefert.
Neben PDF können auch die Formate JPEG, JPEG2000, JBIG und PNG benutzt werden.

Mit dem \type{\placefigure} Befehl werden aus den Bildern Gleitobjekte.
Außerdem werden sie ins Abbildungsverzeichnis aufgenommen.
Die Standardplatzierungsoption in ConTeXt ist \type{[here]}.
In einigen Fällen muss sie explizit angegeben werden, 
da sonst der Text {\bf Abbildung} nicht erscheint.
Die Option \type{[none]} unterdrückt die Nummerierung und die Bildunterschrift.

\starttyping
% Ohne Unterschrift und Nummer
\placefigure[none]{Noch eine Kuh am Berg}{
  \externalfigure[cow.pdf][orientation=-90,height=3cm]}

% Nur Abbildung und Nummer
\placefigure[here]{}{
  \externalfigure[cow.pdf][orientation=-90,height=3cm]}
\stoptyping

\placefigure[here]{%
\useURL[url5][http://wiki.contextgarden.net/Using_Graphics][][Eine Kuh im Internet]\from[url5]}{\externalfigure[cow.pdf][height=4cm]}

\startsubsection[title={Kuhakrobatik}]

Bilder lassen sich in 90° Schritten rotieren, dazu dient die Option \type{orientation}.

\starttyping
\placefigure[here]{Kuh am Berghang}{
  \externalfigure[cow.pdf][orientation=-90]}% -90, 90, 180, 270
\stoptyping

\placefigure[here]{Kuh am Berghang}{
\externalfigure[cow.pdf][orientation=-90,height=3cm]}

Möchte man die Bildunterschrift mitdrehen, so ist dies als Option dem \type{\placefigure}-Befehl mitzuteilen.
Auch hier sind nur 90° Schritte möglich.

\starttyping
\placefigure[90]{Eine gedrehte Kuh}{
\externalfigure[cow.pdf][height=3cm]}
\stoptyping

\placefigure[90]{Eine gedrehte Kuh}{
\externalfigure[cow.pdf][height=3cm]}

Soll nur die Bildunterschrift gedreht werden, dreht man das Bild wieder zurück.

\starttyping
\placefigure[90]{Eine Kuh mit gedrehter Bildunterschrift}{
\externalfigure[cow.pdf][orientation=-90,height=3cm]}
\stoptyping

\placefigure[90,frame]{Eine Kuh mit gedrehter Bildunterschrift}{
\externalfigure[cow.pdf][orientation=-90,height=3cm]}

Andere Drehwinkel für das Bild lassen sich mit den \type{\rotate}-Befehl erreichen.

\starttyping
\placefigure[here]{Eine um 45° gedrehte Kuh}{
\rotate[rotation=45]{\externalfigure[cow.pdf]}}
\stoptyping

\placefigure[here]{Eine um 45° gedrehte Kuh}{
\rotate[rotation=45]{\externalfigure[cow.pdf][height=3cm]}}

Das funktioniert auch ohne \type{\placefigure}.
Damit auch diese Kuh mittig in der Zeile steht, 
wird sie in eine \type{alignment}-Umgebung eingebettet.
Hier ist zusätzlich noch der Befehl \type{\dontleavehmode} erforderlich.

\starttyping
\startalignment[middle]
\dontleavehmode
\rotate[rotation=45]{\externalfigure[cow.pdf][height=3cm]}
\stopalignment\stoptyping

\startalignment[middle]
\dontleavehmode
\rotate[rotation=45]{\externalfigure[cow.pdf][height=3cm]}
\stopalignment

\stopsubsection

\placefloats
\startsubsection[title={Zuschneiden}]

Manchmal muss man Bilder an den Rändern beschneiden.

\starttyping
\clip[width=2cm, height=3cm, hoffset=0mm, voffset=0mm]
     {\externalfigure[cow.pdf]}
\stoptyping

Ein Ausschnitt von 2 cm \times 3 cm zeigt den Kopf der Kuh.
Der Offset wird dabei von links oben gezählt.

\clip[width=2cm, height=3cm, hoffset=0mm, voffset=0mm]
     {\externalfigure[cow.pdf]}

Man kann ein Bild auch in mehrere Kacheln zerschneiden und diese dann
einzeln anzeigen.
Das benötigt man wenn zwei Schauspieler in ein Kuh-Kostüm einsteigen sollen.

\starttyping
\hbox{
\clip[nx=2,x=1]{\externalfigure[cow.pdf]}%
Schauspieler
\clip[nx=2,x=2]{\externalfigure[cow.pdf]}
}
\stoptyping

\hbox{
\clip[nx=2,x=1]{\externalfigure[cow.pdf]}%
Schauspieler%
\clip[nx=2,x=2]{\externalfigure[cow.pdf]}
}

Dieser Befehl teilt das Bild in X-Richtung in zwei Teile und gibt den ersten bzw. zweiten Teil aus.
Mit \type{ny=} wird das Bild in y-Richtung geteilt.

Der \type{\clip} Befehl funktioniert nicht nur bei Bildern.
Es können z.\,B. auch Texte beschnitten werden.

\blank
\scale[sx=2, sy=3]{\clip[nx=1, ny=2, x=1, y=2]{Geheimtext}}

\stopsubsection


\startsubsection[title={Bildersuche}]

Wo sucht ConTeXt nach Bildern?

Standardmäßig sucht ConTeXt im aktuellen Verzeichnis sowie im darüberliegenden Verzeichnis.
Möchte man die Bilder in ein Unterverzeichnis verbannen, 
so kann man entweder jedes Mal den Pfad angeben:

\starttyping
\externalfigure{Bilder/cow.pdf}
\stoptyping

Einfacher ist es den Pfad für das gesamte Dokument als Suchpfad zu definieren.

\starttyping
\setupexternalfigures[directory={Bilder,../Bilder}] 
\stoptyping

So könnte man auch ein Verzeichnis außerhalb des Dokumentpfades angeben,
dies wird dann aber nicht in die Versionsverwaltung einbezogen und bei der
Weitergabe des Projekts kann man es leicht vergessen.

\starttyping
\setupexternalfigures[directory={/Users/axel/Pictures/context-bilder}] 
\stoptyping

Hierbei gibt es einen Stolperstein: Viele Betriebssysteme verwenden standardmäßig englische
Bezeichnungen für die Verzeichnisse, zeigen auf der grafischen Oberfläche aber übersetzte Namen an.
Den wirklichen Verzeichnisnamen verrät dann das Terminal oder die Eingabeaufforderung. 

\stopsubsection

\startsubsection[title={Farben}]
\index{Farben}

Einige Farben sind in ConTeXt bereits vordefiniert, anderen können über Module hinzugeladen werden.

Ein wichtiger Farbraum ist z.\,B. RAL:

\starttyping
\usemodule[colo-imp-ral]
\color[RAL6011]{Resedagrün} 
\stoptyping

\usemodule[colo-imp-ral]
\color[RAL6011]{Resedagrün} 

Die vordefinierten Farben sind:

\color [red]           {red}
\color [green]         {green}
\color [blue]          {blue}

\color [cyan]          {cyan}
\color [magenta]       {magenta}
\color [yellow]        {magenta}

\color [white]        {white}
\color [black]        {black}
\color [gray]         {gray}

\color [lightred]     {lightred}
\color [lightgreen]   {lightgreen}
\color [lightblue]    {lightblue}
\color [lightcyan]    {lightcyan}
\color [lightmagenta] {lightmagenta}
\color [lightyellow]  {lightyellow}

\color [middlered]    {middlered}
\color [middlegreen]  {middlegreen}
\color [middleblue]   {middleblue}
\color [middlecyan]   {middlecyan}
\color [middlemagenta]{middlemagenta}
\color [middleyellow] {middleyellow}

\color [darkred]      {darkred}
\color [darkgreen]    {darkgreen}
\color [darkblue]     {darkblue}
\color [darkcyan]     {darkcyan}
\color [darkmagenta]  {darkmagenta}
\color [darkyellow]   {darkyellow}

\color [darkgray]     {darkgray}
\color [middlegray]   {middlegray}
\color [lightgray]    {lightgray}
\color [palegray]     {palegray}

\color [orange]       {orange}
\color [lightorange]  {lightorange}
\color [middleorange] {middleorange}
\color [darkorange]   {darkorange}

Außerdem kann man sich eigene Farben passend zur Corporate Identity definieren.
Die angegebene Werte sind \type{RBG}-Werte von 0 bis 1 bzw. ein sechsstelliger Hex-Wert.
So lassen sich in ConTeXt die gleichen Farben verwenden wie in HTML. (YCMV)\footnote{Your Color May Vary.}
Alternativ lassen sich auch Farben in den Farbmodellen \type{CMYK}, \type{HSB} definieren.

\starttyping
\definecolor[ciblue][r=0.000, g=0.314, b=0.580]
\definecolor[salmon][h=AB5757]
\stoptyping

\definecolor [ciblue] [r=0.000, g=0.314, b=0.580]

Für offizielle Dokumente bitte die Farbe \color[ciblue]{ciblue} verwenden.

\definecolor[salmon][h=AB5757]
Ist das wirklich \color[salmon]{lachsfarben}?

Farben können auch transparent sein:

\starttyping
\definecolor[labelbackground][a=1,t=0.2,r=1,g=1,b=0]
\stoptyping

Dabei gibt der Wert \type{a} an, wie die Transparenz mit der Grundfarbe verrechnet wird.

\starttyping
\definetransparency [none]        [0]
\definetransparency [normal]      [1]
\definetransparency [multiply]    [2]
\definetransparency [screen]      [3]
\definetransparency [overlay]     [4]
\definetransparency [softlight]   [5]
\definetransparency [hardlight]   [6]
\definetransparency [colordodge]  [7]
\definetransparency [colorburn]   [8]
\definetransparency [darken]      [9]
\definetransparency [lighten]    [10]
\definetransparency [difference] [11]
\definetransparency [exclusion]  [12]
\stoptyping

Transparente Farben sind vor allem für Graphiken sinnvoll.

\stopsubsection

%\placelistoffigures[criterium=all]

\stopsection

\stopcomponent

