% !TEX TS-program = ConTeXt (LuaTeX 1.0.9)
% !TEX encoding = UTF-8 Unicode

% Copyright 2018 Axel Kielhorn
% Lizenz: CC-BY-SA 4.0 Unported http://creativecommons.org/licenses/by-sa/4.0/deed.de

\startcomponent *
\product prd_intro
\project project_intro

\startsection[
  reference=textausrichtung,
  title={Textausrichtung}
  ]

In manchen Fällen ist es sinnvoll, Text nicht -- wie in ConTeXt üblich --
als Blocksatz zu schreiben, sondern rechtsbündig, linksbündig oder
zentriert.

Dieser Text wird als Blocksatz dargestellt, was man sieht, da er über
mehrere Zeilen geht.

Die Ausrichtung wird mit dem Befehl
\starttyping
\startalignment[flushright]
...
\stopalignment
\stoptyping
geändert.

\index{alignment}
\startalignment[flushleft]
\type{flushleft} Dieser Text wird linksbündig dargestellt und nicht als Blocksatz, was
man sieht, da er über mehrere Zeilen geht.
\stopalignment

\startalignment[flushright]
\type{flushright} Dieser Text wird rechtsbündig dargestellt und nicht als Blocksatz, was
man sieht, da er über mehrere Zeilen geht.
\stopalignment

\startalignment[middle]
\type{middle} Dieser Text wird zentriert dargestellt und nicht als Blocksatz, was man
sieht, da er über mehrere Zeilen geht.
\stopalignment

\stopsection

\stopcomponent
