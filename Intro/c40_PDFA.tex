% !TEX TS-program = ConTeXt (LMTX2022)
% !TEX encoding = UTF-8 Unicode

% Copyright 2018 Axel Kielhorn
% Lizenz: CC-BY-SA 4.0 Unported http://creativecommons.org/licenses/by-sa/4.0/deed.de

\startcomponent *
\product prd_intro
\project project_intro


\startsection[
  reference=pdfa,
  title={PDF/A},
  ]
\index{PDF/A}

Einer der wichtigsten Gründe ConTeXt zu benutzen, ist die Möglichkeit PDF/A zu erstellen.
Dieser für Archivierung vorgesehene Standard wir von vielen Organisationen gefordert 
und kann mit LaTeX nur bedingt erreicht werden.

ConTeXt bietet die Möglichkeit tagged PDF zu erstellen und erreich somit PDF/A-1b.
Im Dokument muss dazu die \type{\start...}/\type{\stop...} Version für Umgebungen verwendet werden.

Außerdem sind folgende Einstellungen erforderlich:

\starttyping
\setupinteraction
  [title=TITLE,
   subtitle=SUBTITLE,
   author=AUTHOR,
   keyword={{KEYWORD1, KEYWORD2}, KEYWORD3}]

%% For PDF/A
\setupbackend[
format={pdf/a-1b:2005}, % oder pdf/a-1a:2005
profile={default_cmyk.icc,default_rgb.icc,default_gray.icc},
intent=ISO coated v2 300\letterpercent\space (ECI)]

%% Tagged PDF
%% method=auto ==> default tags by Adobe
\setupbackend[export=yes]
\setupstructure[state=start,method=auto]
\stoptyping

Außerdem müssen die entsprechenden Profile in \type{tex/texmf-context/colors/icc/profiles}
installiert werden.

\starttyping
default_cmyk.icc
default_gray.icc
default_rgb.icc
ISOcoated_v2_300_eci.icc
\stoptyping

Die \type{default} Profile werden mit \type{Ghostscript} ausgeliefert.
Das \type{ISOcoated} Profil wird mit ConTeXt mitgeliefert.

\startsubsection[title={Probleme}]

Der \type{\chemical} Befehl funktioniert nicht im Text.

\stopsubsection

\stopsection

\stopcomponent

