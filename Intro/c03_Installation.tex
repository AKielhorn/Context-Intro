% !TEX TS-program = ConTeXt (LuaTeX 1.0.9)
% !TEX encoding = UTF-8 Unicode

% Copyright 2018 Axel Kielhorn
% Lizenz: CC-BY-SA 4.0 Unported http://creativecommons.org/licenses/by-sa/4.0/deed.de

\startcomponent *
\product prd_intro
\project project_intro

\startsection[title={Installation}]
\index{Installation}
Installation unter Unix:

\starttyping
mkdir ~/context
cd ~/context
wget http://minimals.contextgarden.net/setup/first-setup.sh

 # Install the latest beta of ConTeXt
 # Flags you can add to the first-setup.sh call:
 #     --modules=all      # Install all third-party modules
 #     --context=current  # Install latest stable version

sh ./first-setup.sh --modules=all
\stoptyping

Dieser Befehl installiert ConTeXT MkIV.
Der Speicherbedarf liegt bei 370 MB.

Die ältere Version (MkII) sollte nicht mehr benutzt werden.
Sie benötigt zusätzlich noch \type{Ruby} und \type{pdftex} bzw.
\type{XeTeX}. Diese Programme werden hier nicht mitinstalliert.

Auf dem Mac gibt es normalerweise kein \type{wget}, 
stattdessen kann die Datei mit \type{rsync} geladen werden.

\starttyping
rsync -av rsync://contextgarden.net/minimals/setup/first-setup.sh .
\stoptyping

Im \goto{Contextgarden}[url(https://wiki.contextgarden.net/ConTeXt_Standalone\#Windows)]
gibt es ein Archiv mit der notwendigen Software für Windows: \index{Windows}

\starttyping
context-setup-mswin.zip
context-setup-win64.zip
\stoptyping

Das Archiv enthält \type{luatex} und \type{rsync}.
Damit lässt sich dann

\starttyping
first-setup.bat --modules=all
\stoptyping

aufrufen. Die Installation dauert abhängig von der Internetverbindung ca. 10 
Minuten.
Dank \type{rsync} werden bei einem Update nur die geänderten Dateien übertragen.

Nach Abschluss der Installation wird folgende Meldung ausgegeben:

\starttyping
When you want to use context, you need to initialize the tree by typing:

  . /Users/axel/context/tex/setuptex

in your shell or add
  "/Users/axel/context/tex/texmf-osx-64/bin"
to PATH variable if you want to set it permanently.
\stoptyping

Wenn man nur eine ConTeXt-Version installiert, 
ist es am einfachsten den \type{PATH} entsprechend zu setzen.

Möchte man mehrere Versionen installieren um eine stabile Version als
Reserve zu haben, gleichzeitig aber die aktuellen Beta-Version auszuprobieren,
kann über den Befehl \type{setuptex} die jeweils gewünschte Version gewählt werden.

Zum Übersetzen wird ConTeXt in der Kommandozeile mit

\starttyping
context dateiname.tex
\stoptyping

aufgerufen. Wenn es keine Fehler gibt, entsteht so eine PDF-Datei sowie ein paar Hilfsdateien.

\startsubsection[title={LuaMetaTeX, die Zukunft}]

Im April 2019 wurde LuaMetaTeX (kurz LMTX) vorgestellt. 
Diese neue, für ConTeXt optimierte, Engine bildet die Basis für zukünftige Entwicklungen von ConTeXt.

Obwohl die Kompatibilität zu MkIV angestrebt wird, gibt es doch ein paar Abweichungen.
In der Regel sind diese aber für den normalen Nutzer nicht relevant.

Mit dem Befehl 
\starttyping
context --luatex
\stoptyping

steht außerdem die alte LuaTeX-basierte Version weiterhin zur Verfügung.

Auf der Seite \goto{LMTX}[url(https://wiki.contextgarden.net/LMTX)] im Context Wiki gibt es weitere Informationen.
Außerdem gibt es eine Installationsanleitung.
Da sich LMTX zur Zeit (Mitte 2021) sehr schnell entwickelt, kann es zu Fehlern kommen,
die in der Regel aber innerhalb weniger Tage korrigiert werden.

Wer LMTX produktiv einsätzen möchte, sollte auf jeden Fall eine stabile Version als Backup bereithalten.
Außerdem ist es empfehlenswert die \goto{Mailing Liste ntg-context}[url(https://wiki.contextgarden.net/Mailing_Lists)] zu abonnieren.
Diese Liste ist englischsprachig.

\stopsubsection

\startsubsection[title={Schreibwerkzeuge}]
\index{Schreibwerkzeug, Editor}

ConTeXt Dateien sind reine Textdateien, d.\,h. sie können mit jedem beliebigen Editor geschrieben werden.
Neben den Klassikern \goto{Vim}[url(https://www.vim.org)] und \goto{Emacs}[url(https://https://www.gnu.org/software/emacs/)],
die selbstverständlich eine Unterstützung für ConTeXt bieten, gibt es zwei weitere:

\goto{SciTE}[url(https://www.scintilla.org)] läuft nur unter Windows, 
\goto{Textadept}[url(https://foicica.com/textadept/)] gibt es auch für andere Betriebssysteme.

Programme wie \type{TeXShop}, \type{TeXworks} und \type{TeXstudio} bieten nur eine Unterstützung für LaTeX,
sind daher weniger hilfreich.

\stopsubsection

\stopsection

\stopcomponent
