% !TEX TS-program = ConTeXt (LuaTeX 1.0.9)
% !TEX encoding = UTF-8 Unicode

% Copyright 2018 Axel Kielhorn
% Lizenz: CC-BY-SA 4.0 Unported http://creativecommons.org/licenses/by-sa/4.0/deed.de

\startcomponent *
\product prd_intro
\project project_intro

\startsection[title={Installation}]
\index{Installation}
Installation unter Unix:

\starttyping
mkdir ~/context
cd ~/context
wget http://minimals.contextgarden.net/setup/first-setup.sh

 # Install the latest beta of ConTeXt
 # Flags you can add to the first-setup.sh call:
 #     --modules=all      # Install all third-party modules
 #     --context=current  # Install latest stable version

sh ./first-setup.sh --modules=all
\stoptyping

Dieser Befehl installiert ConTeXT MkIV.
Der Speicherbedarf liegt bei 370 MB.

Die ältere Version (MkII) sollte nicht mehr benutzt werden.
Sie benötigt zusätzlich noch \type{Ruby} und \type{pdftex} bzw.
\type{XeTeX}. Diese Programme werden hier nicht mitinstalliert.

Auf dem Mac gibt es normalerweise kein \type{wget}, 
stattdessen kann die Datei mit \type{rsync} geladen werden.

\starttyping
rsync -av rsync://contextgarden.net/minimals/setup/first-setup.sh .
\stoptyping

Im \goto{Contextgarden}[url(https://wiki.contextgarden.net/ConTeXt_Standalone\#Windows)]
gibt es ein Archiv mit der notwendigen Software für Windows: \index{Windows}

\starttyping
context-setup-mswin.zip
context-setup-win64.zip
\stoptyping

Das Archiv enthält \type{luatex} und \type{rsync}.
Damit lässt sich dann

\starttyping
first-setup.bat --modules=all
\stoptyping

aufrufen. Die Installation dauert abhängig von der Internetverbindung ca. 10 
Minuten.
Dank \type{rsync} werden bei einem Update nur die geänderten Dateien übertragen.

Zum Übersetzen wird ConTeXt in der Kommandozeile mit

\starttyping
context dateiname.tex
\stoptyping

aufgerufen. Wenn es keine Fehler gibt, entsteht so eine PDF-Datei sowie ein paar Hilfsdateien.

\stopsection

\stopcomponent
