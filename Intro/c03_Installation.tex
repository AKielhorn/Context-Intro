% !TEX TS-program = ConTeXt (LuaTeX 1.0.9)
% !TEX encoding = UTF-8 Unicode

% Copyright 2018 Axel Kielhorn
% Lizenz: CC-BY-SA 4.0 Unported http://creativecommons.org/licenses/by-sa/4.0/deed.de

\startcomponent *
\product prd_intro
\project project_intro

\startsection[title={Installation}]
\index{Installation}
Installation unter Unix und macOS:

\starttyping
mkdir ~/context
cd ~/context
wget http://minimals.contextgarden.net/setup/first-setup.sh

 # Install the latest beta of ConTeXt
 # This takes a long time, so go have a coffee
 # Flags you can add to the first-setup.sh call:
 #     --modules=all      # Install all third-party modules
 #     --context=current  # Install latest stable version
 #     --engine=luatex    # Install only MkIV, leave out MkII
 #                        # This shrinks the install from 
 #                        # 360 MB to 300 MB

sh ./first-setup.sh --engine=luatex --modules=all
\stoptyping

\useURL[contextwindows][https://wiki.contextgarden.net/ConTeXt_Standalone\#Windows][][Contextgarden]

Im \from[contextwindows] gibt es ein Archiv mit der notwendigen Software für Windows:
\index{Windows}

\starttyping
context-setup-mswin.zip
context-setup-win64.zip
\stoptyping

Das Archiv enthält \type{luatex} und \type{rsync}.
Damit lässt sich dann

\starttyping
first-setup.bat --engine=luatex --modules=all
\stoptyping

aufrufen. Die Installation dauert abhängig von der Internetverbindung ca. 10 
Minuten.
Dank \type{rsync} werden bei einem Update nur die geänderten Dateien übertragen.

Zum Übersetzen wird ConTeXt in der Kommandozeile mit

\starttyping
context dateiname.tex
\stoptyping

aufgerufen. Wenn es keine Fehler gibt, entsteht so eine PDF-Datei sowie ein paar Hilfsdateien.

\stopsection

\stopcomponent
