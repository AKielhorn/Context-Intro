% !TEX TS-program = ConTeXt (LMTX2022)
% !TEX encoding = UTF-8 Unicode

% Copyright 2018 Axel Kielhorn
% Lizenz: CC-BY-SA 4.0 Unported http://creativecommons.org/licenses/by-sa/4.0/deed.de

\startcomponent *
\product prd_intro
\project project_intro

\startsection[
  reference=beschreibungslisten,
  title={Beschreibungslisten},
  ]

Eine Beschreibungsliste ist ähnliche einer Liste, hat anstelle der
Listenpunkte aber einen beschreibenden Text. Erstreckt sich der Text über
mehrere Zeilen, werden die folgenden eingerückt.

\index{Beschreibungslisten}

\starttyping
\startdescription{Erde}
  Mostly harmless.
\stopdescription

\startdescription{Menschen}
  (Homo sapiens) sind nach der ...
\stopdescription

\startdescription{Donald E. Knuth}
  (* 10. Januar 1938 in Milwaukee, Wisconsin) ...
\stopdescription
\stoptyping

\startdescription{Erde}
  Mostly harmless.
\stopdescription

\startdescription{Menschen}
  (Homo sapiens) sind nach der biologischen Systematik höhere Säugetiere
  aus der Ordnung der Primaten (Primates). Der Mensch gehört zur
  Unterordnung der Trockennasenaffen (Haplorrhini) und dort zur Familie
  der Menschenaffen (Hominidae).
\stopdescription

\startdescription{Donald E. Knuth}
  (* 10. Januar 1938 in Milwaukee, Wisconsin) ist ein US-amerikanischer
  Informatiker, emeritierter Professor an der Stanford University, Autor
  des Standardwerks The Art of Computer Programming und Urvater des
  Textsatzsystems TeX.
\stopdescription

In der Environment-Datei wurde ein weiteres Layout definiert:

\startldescription{Erde}
Mostly harmless.
\stopldescription

\startldescription{Menschen}
(Homo sapiens) sind nach der biologischen Systematik höhere Säugetiere
aus der Ordnung der Primaten (Primates). Der Mensch gehört zur
Unterordnung der Trockennasenaffen (Haplorrhini) und dort zur Familie
der Menschenaffen (Hominidae).
\stopldescription

\startldescription{Donald E. Knuth}
(* 10. Januar 1938 in Milwaukee, Wisconsin) ist ein US-amerikanischer
Informatiker, emeritierter Professor an der Stanford University, Autor
des Standardwerks The Art of Computer Programming und Urvater des
Textsatzsystems TeX.
\stopldescription
\stopsection

\stopcomponent

