% !TEX TS-program = ConTeXt (LuaTeX 1.0.9)
% !TEX encoding = UTF-8 Unicode

% Copyright 2018 Axel Kielhorn
% Lizenz: CC-BY-SA 4.0 Unported http://creativecommons.org/licenses/by-sa/4.0/deed.de

\startcomponent *
\product prd_intro
\project project_intro

\startsection[
  reference=trennungen,
  title={Trennungen}
  ]

Auch wenn der Trennalgorithmus von TeX nicht schlecht ist, gibt es
wieder wieder Wörter, die man gerne anders oder gar nicht trennen würde.
Aus dem Grund gibt es verschiedene Möglichkeit, Worttrennungen zu
forcieren bzw. zu empfehlen.

\startitemize[margin=4cm]
\item
  Dies ist ein etwas längerer Text, der das Wort „Trennalgorithmus“
  trennen soll, wie es LaTeX normalerweise trennen würde. (Und zumindest
  bei mir trennt TeX fehlerhaft mit „Tren-nalgorithmus“.)
\item
  Dies ist ein etwas längerer Text, der das Wort „Trenn\-algorithmus“
  trennen soll. Hier wurde die Trennung mit \type{\-} forciert.
\item
  Hier etwas  Fülltext als zuvor, der das Wort „Floss/Libre“ trennen
  soll. Es ragt über den Rand.
\stopitemize

Der oben beschrieben Fehler tritt mit den aktuellen Trennmustern nicht mehr auf.

\stopsection

\stopcomponent
