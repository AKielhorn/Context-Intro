% !TEX TS-program = ConTeXt (LMTX2022)
% !TEX encoding = UTF-8 Unicode

% Copyright 2018 Axel Kielhorn
% Lizenz: CC-BY-SA 4.0 Unported http://creativecommons.org/licenses/by-sa/4.0/deed.de

\startcomponent *
\product prd_intro
\project project_intro

\startsection[
  reference=trennungen,
  title={Trennungen},
  ]

Auch wenn der Trennalgorithmus von TeX nicht schlecht ist, gibt es
wieder wieder Wörter, die man gerne anders oder gar nicht trennen würde.
Aus dem Grund gibt es verschiedene Möglichkeit, Worttrennungen zu
forcieren bzw. zu empfehlen.

\startitemize[margin=4cm]
\item
  Dies ist ein etwas längerer Text, der das Wort „Trennalgorithmus“
  trennen soll, wie es LaTeX normalerweise trennen würde. (Und zumindest
  bei mir trennt TeX fehlerhaft mit „Tren-nalgorithmus“.)
\item
  Dies ist ein etwas längerer Text, der das Wort „Trenn\-algorithmus“
  trennen soll. Hier wurde die Trennung mit \tex{-} forciert.
\item
  Hier etwas  Fülltext als zuvor, der das Wort „Floss/Libre“ trennen
  soll. Es ragt über den Rand.
\stopitemize

Der oben beschrieben Fehler tritt mit den aktuellen Trennmustern nicht mehr auf.

Bei schmalen Spalten oder langen nichttrennbaren Wörter kann es trotzdem zu übervollen Zeilen kommen.
Hier sollte man ConTeXt etwas mehr Leerraum gönnen.
Der Befehl \tex{setupalign} hilft hier weiter:

\starttyping
\setupalign[hyphenated,morehyphenation,flushleft]
\stoptyping

Die möglichen Optionen sind:

\starttabulate[|T|l|]
\NC hanging         \NC Satzzeichen hängen in den Rand \NR
\NC nohanging       \NC Satzzeichen hängen nicht in den Rand \NR
\NC hz              \NC Buchstaben dürfen leicht verbreitert werden \NR
\NC nohz            \NC Buchstaben dürfen nicht verbreitert werden \NR
\NC hyphenated      \NC Mit Trennung \NR
\NC nothyphenated   \NC Ohne Trennung \NR
\NC lesshyphenation \NC Weniger Trennstellen \NR
\NC morehyohenation \NC Mehr Trennstellen \NR
\NC tolerant        \NC Mehr Wortzwischenraum \NR
\NC verytolerant    \NC Noch mehr Wortzwischenraum \NR
\NC stretch         \NC Riesige Wortzwischenräume \NR
\NC flushleft       \NC Linksbündig ohne Blocksatz \NR
\NC flushright      \NC Rechtsbündig ohne Blocksatz \NR
\NC middle          \NC Horizontal zentrierter Satz \NR
\stoptabulate

Man kann die Einstellung auch nur für bestimmte Absätze ändern, 
z.\,B. wenn viele Wörter mit \tex{type} gesätzt werden.

\starttyping
\startalignment[verytolerant]
Nur hier geändert
\stopalignment
\stoptyping

\stopsection

\stopcomponent
