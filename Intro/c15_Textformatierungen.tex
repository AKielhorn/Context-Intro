% !TEX TS-program = ConTeXt (LuaTeX 1.0.9)
% !TEX encoding = UTF-8 Unicode

% Copyright 2018 Axel Kielhorn
% Lizenz: CC-BY-SA 4.0 Unported http://creativecommons.org/licenses/by-sa/4.0/deed.de

\startcomponent *
\product prd_intro
\project project_intro

\startsection[
  reference=textformatierungen,
  title={Textformatierungen}
  ]
\index{Kursiv}\index{fett}
In diesem Abschnitt sollen verschiedenen Textformatierungen ausprobiert
werden.

\startitemize
\item
  hervorgehoben: \type{\em} {\em hervorgehoben}
\item
  fett: \type{\bf} {\bf fett}
\item
  kursiv: \type{\em} {\em kursiv}
\item
  geneigt: \type{\sl} {\sl geneigt}
\item
  Sans-Serif: \type{\ss} {\ss Sans Serif}
\item
  Schreibmaschine: \type{\tt} {\tt Schreibmaschine}
\item
  Kapitälchen: \type{\sc} {\sc Kapitälchen}
\item
  unterstrichen 1: {\underbar alles unterstrichen}
\item
  unterstrichen 2: {\underbars Jedes Wort wird einzeln unterstrichen}
\item
  Oldstyle Ziffern: Normale Ziffern 1234567890 oder mit \type{\os} {\os 1234567890}

\stopitemize

Textformatierungen kann man auch blockweise für ganze Absätze nutzen:

{\bf Dies ist ein kurzer Absatz, der mit \type{\bf} fett gedruckt
wurde.}

{\sc Dies ist ein kurzer Absatz, der mit \type{\sc} als Kapitälchen
gesetzt wurde.}

{\sl Dies ist ein kurzer Absatz, der mit \type{\sl} geneigt gesetzt
wurde.}

{\it Dies ist ein kurzer Absatz, der mit \type{\it} kursiv gesetzt
wurde.}

{\tt Dies ist ein kurzer Absatz, der mit \type{\tt} in
Schreibmaschinenschrift gesetzt wurde.}

{\ss Dies ist ein kurzer Absatz, der mit \type{\ss} serifenlos gesetzt
wurde.}
\stopsection

\stopcomponent

