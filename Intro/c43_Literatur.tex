% !TEX TS-program = ConTeXt (LMTX2022)
% !TEX encoding = UTF-8 Unicode

% Copyright 2018 Axel Kielhorn
% Lizenz: CC-BY-SA 4.0 Unported http://creativecommons.org/licenses/by-sa/4.0/deed.de

\startcomponent *
\product prd_intro
\project project_intro

\startsection[
  reference=Literatur,
  title={Literaturverwaltung},
  ]

Es steht geschrieben, aber wo?

Bei wissenschaftlichen Arbeiten ist es selbstverständlich fremde (oder auch eigene) Werke zu zitieren und dabei die Quelle anzugeben.

Die Quellen sammelt man am besten in einer Datei und überlässt es dann einem Programm die relevanten Informationen rauszusuchen und richtig zu formatieren.
Lange Zeit war BibTeX das dazu verwendete Programm, aber inzwischen kann ConTeXt die Arbeit selbst übernehmen und ist so auf keine externen Hilfsprogramme angewiesen.

Da das BibTeX Datenbankformat weit verbreitet ist, ließt ConTeXt Dateien in diesem Format.
Es gibt mehrere Programme um BibTeX Dateien zu verwalten, 
weit verbreitet ist \goto{jabref}[url(http://jabref.sourceforge.net)].
Ein normaler Editor reicht jedoch vollkommen aus.

Außerdem kann ConTeXt folgende Formate lesen:

\blank[big]
{
\bTABLE
\setupTABLE[frame=off]
\bTR \bTD savedrecs.txt \eTD \bTD Institute of Scientific Information (ISI) tagged format (e.g. Thomson Reuters™ Web of Science™) \eTD \eTR
\bTR \bTD filename.enw  \eTD \bTD Thomson Reuters™ Endnote™ export format (there is also an Endnote .xml export) \eTD \eTR
\bTR \bTD filename.ris  \eTD \bTD Research Information Systems, Incorporated, now Thomson Reuters™ Reference Manager™ \eTD \eTR
\bTR \bTD pubmed_result.txt \eTD \bTD The National Library of Medicine® (NLM®) MEDLINE®/PubMed® data format\eTD \eTR
\eTABLE
}

\startsubsection[title={Das BibTeX Format}]

BibTeX kennt unterschiedliche Dokumentkategorien und erwartet dafür entsprechende Felder.
Die bekannten Dokumentkategorien und die unterstützten Felder zeigt der Befehl

\starttyping
\showbtxfields[rotation=90]
\stoptyping

Dabei werden die Pflichteinträge grün markiert.

{\tfx\showbtxfields[rotation=90]}

Ein Eintrag in der Datenbank hat die Form \type{@kategorie{Schlüssel, Feld = {Wert}}}

Ein Dokument der Kategorie \type{Book} benötigt folgende Felder: \type{author} oder \type{editor}, \type{publisher} und \type{title}.
Das Feld \type{year} ist nicht unbedingt erforderlich, aber empfehlenswert. Eine Zitatstile erzeugen daraus den Text für die Quellenangabe.

Der Eintrag sieht also folgendermaßen aus:

\starttyping
@book{Kielhorn-Context,
  author    = {Axel Kielhorn},
  title     = {ConTeXt Intro},
  publisher = {Läufer Verlag},
  year      = 2018,
}
\stoptyping
\stopsubsection

\startsubsection[title={Datenbanken und Ausgabeformat}]

ConTeXt kann mehrere Datenbanken mit unterschiedlichen Konfigurationen gleichzeitig benutzen.
In dieser Anleitung beschränke ich mich auf eine Datenbank.

\starttyping
\definebtxdataset[default]
\usebtxdataset[default][epub_latex_beispiel.bib]
\usebtxdataset[default][intro.bib]
\usebtxdefinitions[apa]  % Alphabetisches Verzeichnis
%\usebtxdefinitions[aps] % Numerisches Verzeichnis
\stoptyping

Es wird eine Datenbank mit dem Namen \type{default} definiert.
Diese Datenbank ließt ihre Daten aus zwei Dateien vom Typ BibTeX.
Die Kategorien und Felder werden gemäß er \type{apa}\footnote{American Psychological Association} Definition geladen.
Alternativ stehen die Definitionen gemäß \type{aps}\footnote{American Physical Society} zur Verfügung.

Zur Ausgabe des Literaturverzeichnisses dient der Befehl

\starttyping
%\placelistofpublications[apa][method=dataset] % Gesamte Datenbank
\placelistofpublications[apa][method=global]   % Gesamtes Dokument
%\placelistofpublications[apa][method=local]   % Dieser Abschnitt
\stoptyping

\blank[big]
%\placelistofpublications[apa][method=dataset]
\placelistofpublications[apa][method=global]
%\placelistofpublications[apa][method=local]

Die Option \type{apa} gibt an, wie die Daten formatiert werden.
Hier wird z.\,B. beim zweiten und den weiteren Werken eines Autors der Name durch einen Strich ersätzt.

In den Naturwissenschaften wäre \type{aps} der bessere Ausgangspunkt. 
Im Normalfall entspricht die Formatierung nicht den Wünschen, 
das Format lässt sich jedoch leicht ändern.

Mit der Einstellung \type{method=dataset} werden alle in der Datenbank erfassten Einträge ausgegeben.
Das ist sinnvoll um einen Überblick über die Datenbank zu erhalten.
In realen Dokumenten kommt \type{method=global} für ein Gesamtliteraturverzeichnis,
oder \type{method=local} für ein Kapitelliteraturverzeichnis zum Einsatz.

Eine geänderte Formatierung lässt sich leicht definieren, hier wird eine Nummerierung hinzugefügt.
Die Sortierung bleibt aber \type{authoryear}, 
als zu erst die Autoren alphabetisch sortieren, 
dann innerhalb eines Autors nach Erscheinungsjahr.

\starttyping
\definebtxrendering[intro]
[apa]
[dataset=default,repeat=yes]
\setupbtxrendering[intro]
[sorttype=authoryear, % authoryear, short, cite, index
 numbering=yes,       % yes no num tag
]
\stoptyping

\definebtxrendering[intro]
[apa]
[dataset=default,repeat=yes]
\setupbtxrendering[intro]
[sorttype=authoryear, % authoryear, short, cite, index
 numbering=yes,       % yes no num tag
]

\blank[big]
\placelistofpublications[intro][method=dataset]

Für ein nummeriertes Literaturverzeichnis ist \type{aps} der besserer Ausgangspunkt.

\loadbtxdefinitionfile[aps]
\definebtxrendering
  [introaps]
  [aps]
  [dataset=default,
   group=introaps,
   repeat=yes
]
\setupbtxrendering[introaps]
[sorttype=cite, % authoryear, short, cite, index
]

\blank[big]
\placelistofpublications[introaps][method=dataset]

Normalerweise wird eine Quelle nur einmal angegeben.
Für diese Beispiele musste daher zusätzlich die Option \type{repeat=yes} angegeben werden,
damit die zusätzlichen Literaturverzeichnisse Daten enthalten.

%\placelistofpublications[criterium=chapter]

\stopsubsection

\startsubsection[title={Zitieren}]

Zur Quellenangabe im Text wird der Befehl \type{\cite[Schlüssel]} benutzt.
Die Formatierung für \type{apa} sieht so aus \cite[Kielhorn-Ziele-dtk].

Zusätzlich können noch Texte vor und nach der Quellenangabe ausgegeben werden,
z.\,B. eine Seitenzahl \type{\cite[righttext={ S.\nbsp 12}][Kielhorn-Context]}:
\cite[righttext={ S.\nbsp 35}][Kielhorn-Context]
\type{\nbsp} ist dabei ein nicht trennbares Leerzeichen, 
es verhindert, das die Seitenzahl in die nächste Zeile rutscht.

Über Optionen kann auf weitere Daten eines Eintrags zugegriffen werden:

{\bTABLE
\setupTABLE[frame=off]
\setupTABLE[1][width=0.6\textwidth]
\bTR \bTD \type{\cite[Kielhorn-Context] }              \eTD \bTD \cite[Kielhorn-Context]              \eTD \eTR
\bTR \bTD \type{\cite[num][Kielhorn-Context] }         \eTD \bTD \cite[num][Kielhorn-Context]         \eTD \eTR
\bTR \bTD \type{\cite[textnum][Kielhorn-Context] }     \eTD \bTD \cite[textnum][Kielhorn-Context]     \eTD \eTR
\bTR \bTD \type{\cite[authornum][Kielhorn-Context] }   \eTD \bTD \cite[authornum][Kielhorn-Context]   \eTD \eTR
\bTR \bTD \type{\cite[authoryear][Kielhorn-Context] }  \eTD \bTD \cite[authoryear][Kielhorn-Context]  \eTD \eTR
\bTR \bTD \type{\cite[authoryears][Kielhorn-Context] } \eTD \bTD \cite[authoryears][Kielhorn-Context] \eTD \eTR
\bTR \bTD \type{\cite[short][Kielhorn-Context] }       \eTD \bTD \cite[short][Kielhorn-Context]       \eTD \eTR
\bTR \bTD \type{\cite[tag][Kielhorn-Context] }         \eTD \bTD \cite[tag][Kielhorn-Context]         \eTD \eTR
\bTR \bTD \type{\cite[index][Kielhorn-Context] }       \eTD \bTD \cite[index][Kielhorn-Context]       \eTD \eTR
\bTR \bTD \type{\cite[category][Kielhorn-Context] }    \eTD \bTD \cite[category][Kielhorn-Context]    \eTD \eTR
\bTR \bTD \type{\cite[author][Kielhorn-Context] }      \eTD \bTD \cite[author][Kielhorn-Context]      \eTD \eTR
\bTR \bTD \type{\cite[year][Kielhorn-Context] }        \eTD \bTD \cite[year][Kielhorn-Context]        \eTD \eTR
\bTR \bTD \type{\cite[title][Kielhorn-Context] }       \eTD \bTD \cite[title][Kielhorn-Context]       \eTD \eTR
\bTR \bTD \type{\cite[keywords][Kielhorn-Context] }    \eTD \bTD \cite[keywords][Kielhorn-Context]    \eTD \eTR
\bTR \bTD \type{\cite[none][Kielhorn-Context] }        \eTD \bTD \cite[none][Kielhorn-Context]        \eTD \eTR
\bTR \bTD \type{\cite[entry][Kielhorn-Context]}        \eTD \bTD \cite[entry][Kielhorn-Context]       \eTD \eTR
\bTR \bTD \type{}                                      \eTD \bTD                                      \eTD \eTR
\eTABLE
}

%\usecitation[becker2012]
%\nocite[stein2000]


\stopsubsection

\stopcomponent

