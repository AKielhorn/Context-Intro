% !TEX TS-program = ConTeXt (LMTX2022)
% !TEX encoding = UTF-8 Unicode

% Copyright 2018 Axel Kielhorn
% Lizenz: CC-BY-SA 4.0 Unported http://creativecommons.org/licenses/by-sa/4.0/deed.de

\startcomponent *
\product prd_intro
\project project_intro

\startsection[
  reference=aufzählung,
  title={Aufzählung},
  ]
\index{Aufzählung}

Hier ist eine einfach Aufzählung. Die
Liste wird dreifach verschachtelt.

\starttyping
\startitemize[n][stopper=.]
\item
  Listenpunkt 1
\item
  Listenpunkt 2

  \startitemize[n][stopper=.]
  \item
    Listenpunkt 2.1
  \item
    Listenpunkt 2.2
  \stopitemize
\item
  Listenpunkt 3
\item
  Listenpunkt 4

  \startitemize[n][stopper=.]
  \item
    Listenpunkt 4.1
  \item
    Listenpunkt 4.2

    \startitemize[n][stopper=)]
    \item
      Listenpunkt 4.2.1
    \item
      Listenpunkt 4.2.2
    \stopitemize
  \stopitemize
\stopitemize
\stoptyping

\startitemize[n][stopper=.]
\item
  Listenpunkt 1
\item
  Listenpunkt 2

  \startitemize[n][stopper=.]
  \item
    Listenpunkt 2.1
  \item
    Listenpunkt 2.2
  \stopitemize
\item
  Listenpunkt 3
\item
  Listenpunkt 4

  \startitemize[n][stopper=.]
  \item
    Listenpunkt 4.1
  \item
    Listenpunkt 4.2

    \startitemize[n][stopper=)]
    \item
      Listenpunkt 4.2.1
    \item
      Listenpunkt 4.2.2
    \stopitemize
  \stopitemize
\stopitemize

Die Zählweise lässt sich leicht anpassen:

\starttyping
\startitemize[n,repeat][stopper=.]
\item
  Listenpunkt 1
\item
  Listenpunkt 2

  \startitemize[a,repeat][stopper=)]
  \item
    Listenpunkt 2.1
  \item
    Listenpunkt 2.2
  \stopitemize
\item
  Listenpunkt 3
\item
  Listenpunkt 4

  \startitemize[a,repeat][stopper=)]
  \item
    Listenpunkt 4.1
  \item
    Listenpunkt 4.2

    \startitemize[n][stopper=)]
    \item
      Listenpunkt 4.2.1
    \item
      Listenpunkt 4.2.2
    \stopitemize
  \stopitemize
\stopitemize
\stoptyping

\startitemize[n,repeat][stopper=.]
\item
  Listenpunkt 1
\item
  Listenpunkt 2

  \startitemize[a,repeat][stopper=,width=2em]
  \item
    Listenpunkt 2.1
  \item
    Listenpunkt 2.2
  \stopitemize
\item
  Listenpunkt 3
\item
  Listenpunkt 4

  \startitemize[a,repeat][stopper=,width=2em]
  \item
    Listenpunkt 4.1
  \item
    Listenpunkt 4.2

    \startitemize[n][stopper=),width=3em]
    \item
      Listenpunkt 4.2.1
    \item
      Listenpunkt 4.2.2
    \stopitemize
  \stopitemize
\stopitemize

Um den Buchstaben kann man auch Klammern setzen.

\starttyping
\startitemize[a][left=(, right=), stopper=]
\item
  Listenpunkt 1
\item
  Listenpunkt 2
\stopitemize
\stoptyping

\startitemize[a][left=(, right=), stopper=]
\item
  Listenpunkt 1
\item
  Listenpunkt 2
\stopitemize

Bei kurzen Einträgen können diese auf mehrere Spalten verteilt werden.
Dabei werden zuerst die Spalten aufgefüllt. Durch den Parameter \type{joinedup}
wird der Zeilenabstand unterdrückt.

\starttyping
\startitemize[columns,n,joinedup]
\item eins
\item zwei
\item drei
\item eins
\item zwei
\item drei
\stopitemize
\stoptyping

\startitemize[columns,n,joinedup]
\item eins
\item zwei
\item drei
\item eins
\item zwei
\item drei
\stopitemize

Das geht auch zeilenweise, allerdings müssen die Einträge dann in 
\type{\startitem} \type{\stopitem} Befehle eingeschlossen werden.

\starttyping
\startitemize[n,horizontal,three]
\startitem eins \stopitem
\startitem zwei \stopitem
\startitem drei \stopitem
\startitem eins \stopitem
\startitem zwei \stopitem
\startitem drei \stopitem
\stopitemize
\stoptyping

\startitemize[n,horizontal,three]
\startitem eins \stopitem
\startitem zwei \stopitem
\startitem drei \stopitem
\startitem eins \stopitem
\startitem zwei \stopitem
\startitem drei \stopitem
\stopitemize

Damit man das nicht bei jeder Aufzählung angeben muss, definiert man es global für das gesamte Dokument.

\starttyping
% in der env-Datei
\defineitemgroup[aufzählung][level=3]
\setupitemgroup[aufzählung][1][n,repeat][stopper=)]
\setupitemgroup[aufzählung][2][a,repeat][stopper=),width=2em]
\setupitemgroup[aufzählung][3][n][width=3em]

% im Dokument
\startaufzählung
\item
  Listenpunkt 1
\item
  Listenpunkt 2

  \startaufzählung
  \item
    Listenpunkt 2.1
  \item
    Listenpunkt 2.2
      \startaufzählung
        \item
          Listenpunkt 2.2.1
        \item
          Listenpunkt 2.2.2
      \stopaufzählung
  \stopaufzählung
\item
  Listenpunkt 3
\item
  Listenpunkt 4
\stopaufzählung
\stoptyping

\startaufzählung
\item
  Listenpunkt 1
\item
  Listenpunkt 2

  \startaufzählung
  \item
    Listenpunkt 2.1
  \item
    Listenpunkt 2.2
      \startaufzählung
        \item
          Listenpunkt 2.2.1
        \item
          Listenpunkt 2.2.2
      \stopaufzählung
  \stopaufzählung
\item
  Listenpunkt 3
\item
  Listenpunkt 4
\stopaufzählung

Die folgende Aufzählung soll mit römischen Ziffern beginnen.
ConTeXt kann das ohne Zusatzpaket.

\starttyping
\startitemize[R][width=2.0em,itemalign=flushright,stopper={. }]
\item
  Listenpunkt 1
\item
  Listenpunkt 2
\stopitemize
\stoptyping

\startitemize[R][width=2.0em,itemalign=flushright,stopper={. }]
\item
  Listenpunkt 1
\item
  Listenpunkt 2
\stopitemize
\stopsection

\stopcomponent

