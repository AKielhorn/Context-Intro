% !TEX TS-program = ConTeXt (LuaTeX 1.0.9)
% !TEX encoding = UTF-8 Unicode

% Copyright 2018 Axel Kielhorn
% Lizenz: CC-BY-SA 4.0 Unported http://creativecommons.org/licenses/by-sa/4.0/deed.de

\startcomponent *
\product prd_intro
\project project_intro

\startsection[
  reference=Verweise,
  title={Verweise}
  ]

\index{Links}\index{Verweise}
Anstelle des \type{\label} Befehls gibt man die Referenz in eckigen Klammern bei den Gliederungsbefehlen an. 
Benutzt man MkIV Syntax, so wird stattdessen eine Referenz in \type{\startsection} Befehl definiert.

Verweise im Text erreicht man mit \type{\label} und \type{\ref}. So
wurde am Anfang dieses Teils ein Label \type{Beispiele} gesetzt, auf
welches man mit „siehe Teil {[}Beispiele{]}“ verweisen kann. Alternativ
kann man mit \type{\pageref} auf die Seite verweise, also „siehe Seite”.
Das Verweisen klappt auch auf Abschnitte, siehe dazu 
\in{Abschnitt}[Verweise] auf \at{Seite}[Verweise].

\starttyping
\section[whatever]{Who cares}
... 
\stoptyping

\starttyping
\startsection[reference=whatever,title={Who cares}]
... 
\stopsection
\stoptyping

Auf die Referenzen kann man mit verschiedenen Befehlen zugreifen:

\starttyping
\at {page}[other::whatever]
\in {chapter}[other::whatever]
\about [other::whatever]
\goto{location}[other::whatever]
\stoptyping

Es ist möglich für jedes Kapitel einen eigenen Namensraum zu schaffen. Somit können Konflikte bei gleiche Referenzen in unterschiedlichen Kapiteln vermieden werden.

\stopsection

\stopcomponent

