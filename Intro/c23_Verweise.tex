% !TEX TS-program = ConTeXt (LuaTeX 1.0.9)
% !TEX encoding = UTF-8 Unicode

% Copyright 2018 Axel Kielhorn
% Lizenz: CC-BY-SA 4.0 Unported http://creativecommons.org/licenses/by-sa/4.0/deed.de

\startcomponent *
\product prd_intro
\project project_intro

\startsection[
  reference=Verweise,
  title={Verweise}
  ]

\index{Links}\index{Verweise}
Bei Gliederungselementen kann eine \type{reference} angegeben werden.
Damit kann man sich später auf dieses Element beziehen.

\starttyping
\startsection[
  reference=Verweise,
  title={Verweise}
  ]
\stopsection
% oder
\section[Verweis2]{Verweis}
\stoptyping

Eine Reference kann auch im Text auftauchen.

\starttyping
\reference[label1]{Text}
\stoptyping


Auf die Referenzen kann man mit verschiedenen Befehlen zugreifen:

\starttyping
\at   {Seite}        [Verweise]
\in   {Kapitel}      [Verweise]
\about               [Verweise]
\goto {Nur der Link.}[Verweise]
\stoptyping

Auf \at{Seite}[Verweise] im \in{Kapitel}[Verweise] mit dem Namen \about[Verweise]. \goto{Nur der Link.}[Verweise]


Es ist möglich für jedes Kapitel einen eigenen Namensraum zu schaffen.
Somit können Konflikte bei gleiche Referenzen in unterschiedlichen Kapiteln vermieden werden.

\starttyping
\reference[Kapitel1:Einleitung]{Einleitung}
\reference[Kapitel2:Einleitung]{Einleitung}
\stoptyping

\stopsection

\stopcomponent

