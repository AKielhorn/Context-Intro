% !TEX TS-program = ConTeXt (LMTX2021)
% !TEX encoding = UTF-8 Unicode

% Copyright 2018 Axel Kielhorn
% Lizenz: CC-BY-SA 4.0 Unported http://creativecommons.org/licenses/by-sa/4.0/deed.de

\startcomponent *
\product prd_intro
\project project_intro

\startsection[
  reference=Tabellen,
  title={Tabellen},
  ]
\index{Tabellen}
Es gibt mehrere Möglichkeiten in ConTeXt Tabellen zu erzeugen.
Die gängigsten sind:

\startdescription{Tabulate}
Für einfache Tabellen. Dieser Mechanismus wird auch zum Ausrichten von Formeln benutzt.
\stopdescription

\startdescription{Table}
Veraltet, sollte für neue Dokumente nicht mehr verwendet werden.
\stopdescription

\startdescription{Tables}
Veraltet, sollte für neue Dokumente nicht mehr verwendet werden.
\stopdescription

\startdescription{TABLE}
Natürliche Tabellen (natural tables) sehr leistungsfähig, erfordert viel Schreibaufwand.
\stopdescription

\startdescription{xtables}
Extreme Tabellen (extreme tables) weitgehend kompatibel mit natürlichen Tabellen. Bessere Seitenumbrüche.
\stopdescription

\startsubsection[
  reference=Tabulate,
  title={Tabulate},
  ]
\index{Tabulate}
Mit der \type{tabulate} Umgebung lassen sich einfache Tabellen leicht erstellen.

\starttyping
\starttabulate[|r|l|]
\HL
\NC {\bf Format} \NC {\bf Bedeutung}                 \NR
\HL
\NC c           \NC Zentriert                        \NR
\NC l           \NC Linksbündig                      \NR
\NC r           \NC Rechtsbündig                     \NR
\NC w(1cm)      \NC Einzeilig, feste Breite          \NR
\NC p(1cm)      \NC Absatz, feste Breite             \NR
\NC g.          \NC Ausgerichtet am Zeichen \type{.} \NR
\NC m           \NC Mathematikmodus                  \NR
\NC M           \NC abgesetzter Mathematikmodus      \NR
\HL
\stoptabulate
\stoptyping

\starttabulate[|r|l|]
\HL
\NC {\bf Format} \NC {\bf Bedeutung}                 \NR
\HL
\NC c           \NC Zentriert                        \NR
\NC l           \NC Linksbündig                      \NR
\NC r           \NC Rechtsbündig                     \NR
\NC w(1cm)      \NC Einzeilig, feste Breite          \NR
\NC p(1cm)      \NC Absatz, feste Breite             \NR
\NC g.          \NC Ausgerichtet am Zeichen \type{.} \NR
\NC m           \NC Mathematikmodus                  \NR
\NC M           \NC abgesetzter Mathematikmodus      \NR
\HL
\stoptabulate

\starttabulate[|c|l|r|g,|m|M|]
\NC \type{c}  \NC \type{l} \NC \type{r} \NC \type{g,} \NC \type{m}   \NC \type{M}   \NR\HL
\NC c         \NC l        \NC r        \NC 1,234     \NC x^{2}      \NC x^3        \NR
\NC Zentriert \NC links    \NC rechts   \NC 23,5      \NC \int_{x=1} \NC \int_{x=1} \NR
\stoptabulate

Die Spaltendefinition lässt sich außerdem durch das Anhängen von Formatierungsoptionen beeinflussen:

\starttabulate[|rT|l|]
\NC {\bf Option} \NC {\bf Bedeutung} \NR
\NC i            \NC Abstand links \NR
\NC j            \NC Abstand rechts \NR
\NC B            \NC Fett \NR
\NC I            \NC Kursiv \NR
\NC S            \NC Schräg \NR
\NC T            \NC Schreibmaschinenschrift \NR
\stoptabulate

Mit dem Befehl \type{\setuptabulate[split=yes]} können lange Tabellen auf mehrere Seiten verteilt werden.
Beim Umbruch verhalten sich die unterschiedlichen Linien etwas anders.
Normalerweise reicht die \type{\HL} Linie aus. 
Bei wiederholten Tabellenköpfen sollten jedoch \type{\FL} und \type{\LL} benutzt werden.

\starttabulate[|lB|l|]
\NC {\bf Linie} \NC {\bf Beschreibung} \NR
\NC \type{\HL}  \NC Horizontale Linie \NR\HL
\NC \type{\FL}  \NC Erste Linie \NR\FL
\NC \type{\ML}  \NC Mittlere Linie \NR\ML
\NC \type{\TL}  \NC Untere Linie \NR\TL
\NC \type{\TB}  \NC Leerzeile (optionale Höhe in \type{[]} angeben)  \NR\TB
\NC \type{\BL}  \NC Letzte Linie \NR\BL
\NC \type{\LL}  \NC Letzte Linie (Tabellenkopf)\NR\LL
\stoptabulate

Bei geteilten Tabellen lässt sich auch ein Tabellenkopf definieren, 
der über die folgenden Tabellenteile gesetzt wird.

\starttyping
\setuptabulate[split=yes, header=repeat]
\starttabulatehead[]
  \FL
  \NC {\bf Tabellen} \NC {\bf Kopf} \NC \NR
  \LL
\stoptabulatehead
\starttabulate[|r|l|]
  \NC Tabellen \NC Zeile \NC \NR
\stoptabulate
\stoptyping

{
\setuptabulate[split=yes, header=repeat]
\starttabulatehead[]
\FL
\NC {\bf Tabellen} \NC {\bf Kopf} \NC \NR
\LL
\stoptabulatehead
\starttabulate[|r|l|]
\NC Tabellen \NC Zeile \NC \NR
\NC Tabellen \NC Zeile \NC \NR
\NC Tabellen \NC Zeile \NC \NR
\NC Tabellen \NC Zeile \NC \NR
\NC Tabellen \NC Zeile \NC \NR
\NC Tabellen \NC Zeile \NC \NR
\NC Tabellen \NC Zeile \NC \NR
\NC Tabellen \NC Zeile \NC \NR
\NC Tabellen \NC Zeile \NC \NR
\NC Tabellen \NC Zeile \NC \NR
\NC Tabellen \NC Zeile \NC \NR
\NC Tabellen \NC Zeile \NC \NR
\NC Tabellen \NC Zeile \NC \NR
\NC Tabellen \NC Zeile \NC \NR
\NC Tabellen \NC Zeile \NC \NR
\NC Tabellen \NC Zeile \NC \NR
\NC Tabellen \NC Zeile \NC \NR
\NC Tabellen \NC Zeile \NC \NR
\NC Tabellen \NC Zeile \NC \NR
\NC Tabellen \NC Zeile \NC \NR
\NC Tabellen \NC Zeile \NC \NR
\NC Tabellen \NC Zeile \NC \NR
\NC Tabellen \NC Zeile \NC \NR
\NC Tabellen \NC Zeile \NC \NR
\NC Tabellen \NC Zeile \NC \NR
\NC Tabellen \NC Zeile \NC \NR
\NC Tabellen \NC Zeile \NC \NR
\NC Tabellen \NC Zeile \NC \NR
\NC Tabellen \NC Zeile \NC \NR
\stoptabulate
}

\startmode[*lmtx]

Ab \LMTX\ vom September 2021 gibt es die Möglichkeit, in einer \type{tabulate}
Umgebung mehrere Spalten zu einer zusammenzufassen.
Damit können diese auch in der einfachen Tabellenumgebung benutzt werden,
ohne auf \type{TABLE} umsteigen zu müssen.

\starttyping
   \starttabulate[|c|c|c|]
       \NC 1   \NC Zweite Spalte \NC Dritte Spalte \NR
       \NC 2   \NC Zweite        \NC Dritte        \NR
       \NC 3 r \NS[1][r]Zweite und dritte          \NR
       \NC 3 c \NS[1][c]Zweite und dritte          \NR
       \NC 3 l \NS[1][l]Zweite und dritte          \NR
   \stoptabulate
\stoptyping

   \starttabulate[|c|c|c|]
       \NC 1   \NC Zweite Spalte \NC Dritte Spalte \NR
       \NC 2   \NC Zweite        \NC Dritte        \NR
       \NC 3 r \NS[1][r]Zweite und dritte          \NR
       \NC 3 c \NS[1][c]Zweite und dritte          \NR
       \NC 3 l \NS[1][l]Zweite und dritte          \NR
   \stoptabulate
\stopmode


\stopsubsection

\startsubsection[
  reference=TABLE,
  title={TABLE},
  ]
\index{TABLE}

Eine Tabelle ist einfach aufgebaut.
Die \type{\setupTABLE} Befehle bestimmen das Aussehen der Tabelle.
Damit sie nur für diese Tabelle gelten, wird die komplette Tabelle in \type{{}} eingeschlossen.

\blank[big]

\starttyping
{\bTABLE
\setupTABLE[c][1][align=right,frame=off]
\setupTABLE[c][2][align=middle,frame=off]
\setupTABLE[c][3][align=left,frame=off]

\bTR 
  \bTD linksbündig \eTD  \bTD zentriert \eTD \bTD rechtsbündig\eTD
\eTR
\bTR
  \bTD längerer Text \eTD \bTD längerer Text \eTD \bTD längerer Text\eTD
\eTR

\eTABLE}
\stoptyping

\index{TABLE}
{\bTABLE
\setupTABLE[c][1][align=right,frame=off]
\setupTABLE[c][2][align=middle,frame=off]
\setupTABLE[c][3][align=left,frame=off]

\bTR \bTD linksbündig \eTD  \bTD zentriert \eTD \bTD rechtsbündig\eTD \eTR
\bTR \bTD längerer Text \eTD \bTD längerer Text \eTD \bTD längerer Text\eTD\eTR

\eTABLE}
\blank[big]

 Mit Linien sieht man es etwas besser:

\starttyping
\setupTABLE[c][1][align=right]
\setupTABLE[c][2][align=middle]
\setupTABLE[c][3][align=left]
\stoptyping

\blank[big]

{\bTABLE
\setupTABLE[c][1][align=right]
\setupTABLE[c][2][align=middle]
\setupTABLE[c][3][align=left]

\bTR \bTD linksbündig \eTD  \bTD zentriert \eTD \bTD rechtsbündig\eTD \eTR
\bTR \bTD längerer Text \eTD \bTD längerer Text \eTD \bTD längerer Text\eTD\eTR

\eTABLE}


Gibt man eine Spaltenbreite vor, so wird der Text passend umbrochen.

\starttyping
\setupTABLE[c][1][width=3cm]
\setupTABLE[c][2][width=5cm]
\stoptyping

\blank[big]

{\bTABLE
\setupTABLE[c][1][width=3cm]
\setupTABLE[c][2][width=5cm]
\bTR
  \bTD Dies ist ein langer Text der hoffentlich nach 3 cm umbrochen wird.
    Die Ausrichtung ist oben.\eTD
  \bTD Im Gegensatz dazu ein Umbruch nach 5 cm.\eTD
\eTR
\eTABLE}
\blank[big]

Mit der Option \type{nc} werden mehrere Spalten (Column) und mit der Option \type{nr} mehrere Zeilen zu einer Zelle zusammengefasst.
Es sind nur rechteckige Zellen möglich.

(Beispiel von
\goto{\hyphenatedurl{https://en.wikibooks.org/wiki/LaTeX/Tables}}[url(https://en.wikibooks.org/wiki/LaTeX/Tables)]):
\blank[big]

\starttyping
\bTR \bTD[nc=2,align=middle] Item \eTD \bTD\eTD \eTR
\stoptyping

{\bTABLE
\setupTABLE[offset=1mm]
\setupTABLE[c][1][frame=off]
\setupTABLE[c][2][frame=off]
\setupTABLE[c][3][frame=off,aligncharacter=yes,alignmentcharacter={.},align=left]
\setupTABLE[r][1,2,3][topframe=on]
\setupTABLE[r][last][bottomframe=on]
\setupTABLE[c][1,2,3][leftframe=on]
\setupTABLE[c][last][rightframe=on]

\bTR \bTD[nc=2,align=middle] Item \eTD \bTD             \eTD \eTR
\bTR \bTD Animal                  \eTD \bTD Description \eTD \bTD Price (\$) \eTD \eTR
\bTR \bTD Gnat                    \eTD \bTD per gram    \eTD \bTD 13.65      \eTD \eTR
\bTR \bTD                         \eTD \bTD each        \eTD \bTD  0.01      \eTD \eTR
\bTR \bTD Gnu                     \eTD \bTD stuffed     \eTD \bTD 92.50      \eTD \eTR
\bTR \bTD Emu                     \eTD \bTD stuffed     \eTD \bTD 33.33      \eTD \eTR
\bTR \bTD Armadillo               \eTD \bTD frozen      \eTD \bTD 8.99       \eTD \eTR
\eTABLE}
\blank[big]

{\bTABLE
\setupTABLE[frame=off,offset=1pt]
\setupTABLE[r][1][bottomframe=on]
%\setupTABLE[c][last][rightframe=on]
\setupTABLE[c][2,3,4,5,6,7][width=1.2em,align=left]
\setupTABLE[c][8][width=1.6em,align=left]
\bTR \bTD Team              \eTD \bTD P \eTD \bTD W \eTD \bTD D \eTD \bTD L \eTD \bTD F  \eTD \bTD A \eTD \bTD Pts \eTD \eTR
\bTR \bTD Manchester United \eTD \bTD 6 \eTD \bTD 4 \eTD \bTD 0 \eTD \bTD 2 \eTD \bTD 10 \eTD \bTD 5 \eTD \bTD 12  \eTD \eTR
\bTR \bTD Celtic            \eTD \bTD 6 \eTD \bTD 3 \eTD \bTD 0 \eTD \bTD 3 \eTD \bTD 8  \eTD \bTD 9 \eTD \bTD 9   \eTD \eTR
\bTR \bTD Benfica           \eTD \bTD 6 \eTD \bTD 2 \eTD \bTD 1 \eTD \bTD 3 \eTD \bTD 7  \eTD \bTD 8 \eTD \bTD 7   \eTD \eTR
\bTR \bTD FC Copenhagen     \eTD \bTD 6 \eTD \bTD 2 \eTD \bTD 1 \eTD \bTD 3 \eTD \bTD 5  \eTD \bTD 8 \eTD \bTD 7   \eTD \eTR
\eTABLE}
\blank[big]

{\bTABLE
\setupTABLE[frame=off,offset=1pt]
\setupTABLE[c][each][align=lohi,frame=off]
\setupTABLE[c][3][align=right]
\setupTABLE[r][2][topframe=on,rulethickness=1.5pt]
\setupTABLE[r][3,7,10,11][topframe=on]
\bTR \bTD [nc=3,align=middle] Team sheet \eTD \eTR
\bTR \bTD Goalkeeper         \eTD \bTD GK \eTD \bTD Paul Robinson   \eTD \eTR
\bTR \bTD [nr=4] Defenders   \eTD \bTD LB \eTD \bTD Lucus Radebe    \eTD \eTR
\bTR                              \bTD DC \eTD \bTD Michael Duburry \eTD \eTR
\bTR                              \bTD DC \eTD \bTD Dominic Matteo  \eTD \eTR
\bTR                              \bTD RB \eTD \bTD Didier Domi     \eTD \eTR
\bTR \bTD [nr=3] Midfielders \eTD \bTD MC \eTD \bTD David Batty     \eTD \eTR
\bTR                              \bTD MC \eTD \bTD Eirik Bakke     \eTD \eTR
\bTR                              \bTD MC \eTD \bTD Jody Morris     \eTD \eTR
\bTR \bTD Forward            \eTD \bTD FW \eTD \bTD Jamie McMaster  \eTD \eTR
\bTR \bTD [nr=2] Strikers    \eTD \bTD ST \eTD \bTD Alan Smith      \eTD \eTR
\bTR                              \bTD ST \eTD \bTD Mark Viduka     \eTD \eTR
\eTABLE}
\blank[big]

Tabellenzellen können an beliebigen Zeichen ausgerichtet werden, in diesem Fall am Dezimalkomma.
\blank[big]
{
\setupTABLE[row][each][frame=off]
%\setupTABLE[c][1][width=0.1\textwidth]
\setupTABLE[c][2][alignmentcharacter={,},aligncharacter=yes,align=middle]
\setupTABLE[c][3][alignmentcharacter={,},aligncharacter=yes,align=middle]

\bTABLE
    \bTABLEbody
        \bTR \bTD Jahr \eTD \bTD Meine Aktie \eTD \bTD Mitbewerber \eTD \eTR
        \bTR \bTD 2010  \eTD \bTD  120,56     \eTD \bTD 100,5       \eTD \eTR
        \bTR \bTD 2011  \eTD \bTD  130,78     \eTD \bTD  90,78      \eTD \eTR
        \bTR \bTD 2012  \eTD \bTD  140,9      \eTD \bTD  99,2       \eTD \eTR
        \bTR \bTD 2013  \eTD \bTD  120,89     \eTD \bTD 120         \eTD \eTR
    \eTABLEbody
\eTABLE
}
\blank[big]

Normalerweise funktioniert das zuverlässig.
Schwierig wird es, wenn auch in der Titelzeile Zahlen vorkommen, 
da der Ausrichtungsalgorithmus versucht, auch diese am Komma auszurichten.

Die Ausrichtung kann daher für eine Zeile deaktiviert werden.

\starttyping
\bTR [aligncharacter=no] \bTD Jahr  \eTD \bTD WKN 123456  \eTD \bTD WKN A78C90  \eTD \eTR
\stoptyping

\blank[big]

{
\setupTABLE[row][each][frame=off]
%\setupTABLE[c][1][width=0.1\textwidth]
\setupTABLE[c][2][alignmentcharacter={,},aligncharacter=yes,align=middle]
\setupTABLE[c][3][alignmentcharacter={,},aligncharacter=yes,align=middle]

\bTABLE
    \bTABLEbody
        \bTR [aligncharacter=no] \bTD Jahr \eTD \bTD WKN 123456 \eTD \bTD WKN A78C90 \eTD \eTR
        \bTR \bTD 2010                     \eTD \bTD  120,56    \eTD \bTD 100,5      \eTD \eTR
        \bTR \bTD 2011                     \eTD \bTD  130,78    \eTD \bTD  90,78     \eTD \eTR
        \bTR \bTD 2012                     \eTD \bTD  140,9     \eTD \bTD  99,2      \eTD \eTR
        \bTR \bTD 2013                     \eTD \bTD  120,89    \eTD \bTD 120        \eTD \eTR
    \eTABLEbody
\eTABLE
}

Alternativ kann die Kopfzeile auch mit \type{\bTH \eTH} definiert werden.
Dadurch werden die Einträge auch gleich hervorgehoben.

\starttyping
\bTR \bTH Jahr \eTH \bTH WKN 123456 \eTH \bTH WKN A78C90 \eTH \eTR
\stoptyping

{
\setupTABLE[row][each][frame=off]
%\setupTABLE[c][1][width=0.1\textwidth]
\setupTABLE[c][2][alignmentcharacter={,},aligncharacter=yes,align=middle]
\setupTABLE[c][3][alignmentcharacter={,},aligncharacter=yes,align=middle]

\bTABLE
    \bTABLEbody
        \bTR \bTH Jahr \eTH \bTH WKN 123456 \eTH \bTH WKN A78C90 \eTH \eTR
        \bTR \bTD 2010 \eTD \bTD  120,56    \eTD \bTD 100,5      \eTD \eTR
        \bTR \bTD 2011 \eTD \bTD  130,78    \eTD \bTD  90,78     \eTD \eTR
        \bTR \bTD 2012 \eTD \bTD  140,9     \eTD \bTD  99,2      \eTD \eTR
        \bTR \bTD 2013 \eTD \bTD  120,89    \eTD \bTD 120        \eTD \eTR
    \eTABLEbody
\eTABLE
}

Tabellen können auch über mehrere Seiten verlaufen.
Die Option \type{split=yes}, bzw. \type{split=repeat} definiert dabei, ob der Tabellenkopf wiederholt werden soll, oder nicht. Außerdem kann man für die Folgeseiten einen anderen Kopf definieren.

\blank[big]

{\bTABLE[split=repeat]
\setupTABLE[frame=off,offset=1pt]
\setupTABLE[c][1,2][align=left]
\setupTABLE[c][3][align=middle]
% Letzte Zeile
\setupTABLE[r][last]
  [style=bold,
   background=color,
   backgroundcolor=green]
% Kopfzeile
\bTABLEhead
\bTR [align=middle]
\bTH [nc=3] Kopfzeile  \eTH
\eTR
\eTABLEhead
\bTABLEnext 
% Wiederholungszeilen auf weiteren Seiten
\bTR [background=color,backgroundcolor=cyan,align=middle]
  \bTH [nc=3] Fortsetung \eTH
\eTR
\eTABLEnext
\bTABLEbody
\bTR
\bTD 1. Zeile\eTD
\bTD rechts\eTD
\bTD zentriert\eTD
\eTR\bTR
\bTD 2. Zeile\eTD
\bTD rechts\eTD
\bTD zentriert\eTD
\eTR\bTR
\bTD 3. Zeile\eTD
\bTD rechts\eTD
\bTD zentriert\eTD
\eTR\bTR
\bTD 4. Zeile\eTD
\bTD rechts\eTD
\bTD zentriert\eTD
\eTR\bTR
\bTD 5. Zeile\eTD
\bTD rechts\eTD
\bTD zentriert\eTD
\eTR\bTR
\bTD 6. Zeile\eTD
\bTD rechts\eTD
\bTD zentriert\eTD
\eTR\bTR
\bTD 7. Zeile\eTD
\bTD rechts\eTD
\bTD zentriert\eTD
\eTR\bTR
\bTD 8. Zeile\eTD
\bTD rechts\eTD
\bTD zentriert\eTD
\eTR\bTR
\bTD 9. Zeile\eTD
\bTD rechts\eTD
\bTD zentriert\eTD
\eTR\bTR
\bTD 10. Zeile\eTD
\bTD rechts\eTD
\bTD zentriert\eTD
\eTR\bTR
\bTD 11. Zeile\eTD
\bTD rechts\eTD
\bTD zentriert\eTD
\eTR\bTR
\bTD 12. Zeile\eTD
\bTD rechts\eTD
\bTD zentriert\eTD
\eTR\bTR
\bTD 13. Zeile\eTD
\bTD rechts\eTD
\bTD zentriert\eTD
\eTR\bTR
\bTD 14. Zeile\eTD
\bTD rechts\eTD
\bTD zentriert\eTD
\eTR\bTR
\bTD 15. Zeile\eTD
\bTD rechts\eTD
\bTD zentriert\eTD
\eTR\bTR
\bTD 16. Zeile\eTD
\bTD rechts\eTD
\bTD zentriert\eTD
\eTR\bTR
\bTD 17. Zeile\eTD
\bTD rechts\eTD
\bTD zentriert\eTD
\eTR\bTR
\bTD 18. Zeile\eTD
\bTD rechts\eTD
\bTD zentriert\eTD
\eTR\bTR
\bTD 19. Zeile\eTD
\bTD rechts\eTD
\bTD zentriert\eTD
\eTR\bTR
\bTD 20. Zeile\eTD
\bTD rechts\eTD
\bTD zentriert\eTD
\eTR\bTR
\bTD 21. Zeile\eTD
\bTD rechts\eTD
\bTD zentriert\eTD
\eTR\bTR
\bTD 22. Zeile\eTD
\bTD rechts\eTD
\bTD zentriert\eTD
\eTR\bTR
\bTD 23. Zeile\eTD
\bTD rechts\eTD
\bTD zentriert\eTD
\eTR\bTR
\bTD 24. Zeile\eTD
\bTD rechts\eTD
\bTD zentriert\eTD
\eTR\bTR
\bTD 25. Zeile\eTD
\bTD rechts\eTD
\bTD zentriert\eTD
\eTR\bTR
\bTD 26. Zeile\eTD
\bTD rechts\eTD
\bTD zentriert\eTD
\eTR\bTR
\bTD 27. Zeile\eTD
\bTD rechts\eTD
\bTD zentriert\eTD
\eTR\bTR
\bTD 28. Zeile\eTD
\bTD rechts\eTD
\bTD zentriert\eTD
\eTR\bTR
\bTD 29. Zeile\eTD
\bTD rechts\eTD
\bTD zentriert\eTD
\eTR\bTR
\bTD 30. Zeile\eTD
\bTD rechts\eTD
\bTD zentriert\eTD
\eTR\bTR
\bTD 31. Zeile\eTD
\bTD rechts\eTD
\bTD zentriert\eTD
\eTR\bTR
\bTD 32. Zeile\eTD
\bTD rechts\eTD
\bTD zentriert\eTD
\eTR\bTR
\bTD 33. Zeile\eTD
\bTD rechts\eTD
\bTD zentriert\eTD
\eTR\bTR
\bTD 34. Zeile\eTD
\bTD rechts\eTD
\bTD zentriert\eTD
\eTR\bTR
\bTD 35. Zeile\eTD
\bTD rechts\eTD
\bTD zentriert\eTD
\eTR\bTR
\bTD 36. Zeile\eTD
\bTD rechts\eTD
\bTD zentriert\eTD
\eTR\bTR
\bTD 37. Zeile\eTD
\bTD rechts\eTD
\bTD zentriert\eTD
\eTR\bTR
\bTD 38. Zeile\eTD
\bTD rechts\eTD
\bTD zentriert\eTD
\eTR\bTR
\bTD 39. Zeile\eTD
\bTD rechts\eTD
\bTD zentriert\eTD
\eTABLEbody
%
% Fußzeile (kann auch vor body stehen)
%
\bTABLEfoot
\bTR
  \bTC  Fuß 1 \eTC
  \bTC  Fuß 2 \eTC
  \bTC  Fuß 3 \eTC
\eTR
\eTABLEfoot
%
\eTABLE}
\blank[big]

Die Rahmen um die Tabellenzellen dürfen auch runde Ecken haben.

% Runde Ecken 1
%
% Achtung! Leerzeichen vor und nach den {} sind wichtig.
%
{
\setupcolors[state=start] 
%\showstruts
\setupTABLE[background=color,backgroundcolor=yellow]
\setupframed[framecolor=darkgreen]

\defineparagraphs[ThreeCols]

\startThreeCols

{
\setupTABLE[row]   [first][topframe=off]%
\setupTABLE[row]   [last] [bottomframe=off]%
\setupTABLE[column][first][leftframe=off]%
\setupTABLE[column][last] [rightframe=off]%
\framed[strut=no,corner=round,offset=.5\linewidth]
{\bTABLE
      \bTR \bTD test \eTD \bTD test \eTD \eTR
      \bTR \bTD test \eTD \bTD test \eTD \eTR
      \bTR \bTD test \eTD \bTD test \eTD \eTR
\eTABLE}
}
Dünner Rahmen

\nextThreeCols

{
\setupTABLE[row]   [first][topframe=off]
\setupTABLE[row]   [last] [bottomframe=off]
\setupTABLE[column][first][leftframe=off]
\setupTABLE[column][last] [rightframe=off]
\framed[strut=no,corner=round,offset=.5\linewidth,,rulethickness=1mm]
{\bTABLE
        \bTR \bTD test \eTD \bTD test \eTD \eTR
        \bTR \bTD test \eTD \bTD test \eTD \eTR
        \bTR \bTD test \eTD \bTD test \eTD \eTR
\eTABLE}
}
Dicker Rahmen

\nextThreeCols

{
\setupTABLE[row]   [first][topframe=off]
\setupTABLE[row]   [last] [bottomframe=off]
\setupTABLE[column][first][leftframe=off]
\setupTABLE[column][last] [rightframe=off]
\framed[strut=no,corner=round,
    radius=1.4mm,offset=-1mm,,rulethickness=.1mm]
    {\framed[strut=no,corner=round,
            radius=1.5mm,offset=0mm,,rulethickness=1mm,framecolor=white]
        {\bTABLE
            \bTR \bTD test \eTD \bTD test \eTD \eTR
            \bTR \bTD test \eTD \bTD test \eTD \eTR
            \bTR \bTD test \eTD \bTD test \eTD \eTR
        \eTABLE}}
}
Dünner Rahmen mit korrigierter Füllung

\stopThreeCols
}
\blank[big]

Im dritten Beispiel wird ein zusätzlicher weißer Rahmen um die Tabelle gezogen,
dadurch werden die Ecken der Zellfüllung übermalt.

% Runde Ecken 2

Das folgende Beispiel kommt ohne diese Tricks aus und benutzt die Option \type{backgroundcorner}.
Damit lassen sich auch runde Ecken innerhalb der Tabelle erzeugen.

\blank[big]
{
\setupTABLE[framecolor=darkgreen]
\setupTABLE [y]    [first][background=color,backgroundcolor=orange,frame=off,bottomframe=on,topframe=on]
\setupTABLE [first][first][backgroundcorner=2,corner=10,frame=on]
\setupTABLE [last] [first][backgroundcorner=4,corner=12,frame=on]
\setupTABLE [row]  [each] [background=color,backgroundcolor=yellow,frame=on]
\setupTABLE [first][2]    [corner=8]
\setupTABLE [last] [2]    [corner=5]
\setupTABLE [first][last] [corner=7]
\setupTABLE [last] [last] [corner=6]

\bTABLE[frame=off,align=middle]
\bTR \bTD one   \eTD \bTD two    \eTD \bTD three \eTD \eTR
\bTR \bTD first \eTD \bTD second \eTD \bTD third \eTD \eTR
\bTR \bTD alpha \eTD \bTD beta   \eTD \bTD gamma \eTD \eTR
\eTABLE
}
\blank[big]

\useURL[contextTABLE][https://wiki.contextgarden.net/TABLE][][Contextgarden]
Tabellen sind extrem vielfältig, mehr Informationen gibt es im \from[contextTABLE], allerdings nur auf englisch.

Mit dem Befehl \type{\placetable} erhält man ein Gleitobjekt mit einer \type{\caption}, auf die man sich später beziehen kann.

\index{table}
\placetable{Eine sehr einfache Tabelle.}
\starttable[|l|c|r|]
\HL
\NC linksbündig
\NC zentriert
\NC rechtsbündig
\NC\AR
\NC längerer Text
\NC längerer Text
\NC längerer Text
\NC\AR
\HL
\stoptable

\placetable{Eine sehr einfache Tabelle mit Linien.}
\starttable[|l|c|r|]
\HL
\NC linksbündig
\NC zentriert
\NC rechtsbündig
\NC\AR
\HL
\NC längerer Text
\NC längerer Text
\NC längerer Text
\NC\AR
\HL
\stoptable

\stopsubsection

\stopsection

\stopcomponent

