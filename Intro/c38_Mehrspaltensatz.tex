% !TEX TS-program = ConTeXt (LuaTeX 1.0.9)
% !TEX encoding = UTF-8 Unicode

% Copyright 2018 Axel Kielhorn
% Lizenz: CC-BY-SA 4.0 Unported http://creativecommons.org/licenses/by-sa/4.0/deed.de

\startcomponent *
\product prd_intro
\project project_intro


\startsection[
  reference=mehrspaltensatz,
  title={Mehrspaltensatz},
  ]
\index{Mehrspaltensatz}

\definepagecolumns [zwei]
        [n=2,
        distance=5mm]

Bei größeren Papierformaten werden die Zeilen oft zu lang.
Eine mögliche Lösung ist es, den Text auf mehrere Spalten zu verteilen.
In schmalen Spalten wird der Zeilenumbruch schwieriger, 
wie man am Beispiel mit den drei Spalten erkennen kann.

Es gibt mehrere Möglichkeiten für den Mehrspaltensatz.

\starttyping
\setupcolumns[n=2,distance=1cm,align=raggedright]
\startcolumns
Dieser Text wir ...
\stopcolumns
\stoptyping
\index{startcolumns}

erzeugt Spalten mit gleicher Höhe.
Dies wird für mehrspaltige Aufzählungen und den Index benutzt.

\setupcolumns[n=2,distance=1cm,align=raggedright]
\startcolumns
Dieser Text wir auf zwei Spalten verteilt.
Er muss etwas länger werden, damit man das sieht.
Noch reicht es nicht, aber mir wird schon noch was einfallen.
So, das war's.
\stopcolumns

\setupcolumns[n=3,distance=1cm,align=raggedright]
\startcolumns
Dieser Text wir auf drei Spalten verteilt.
Er muss etwas länger werden, damit man das sieht.
Noch reicht es nicht, aber mir wird schon noch was einfallen.
So, das war's.
\stopcolumns

Soll das gesamte Dokument zweispaltig sein, so ist 
\starttyping
\definepagecolumns [zwei]
        [n=2,
        distance=5mm]
\startpagecolumns[zwei]
\setuptolerance[tolerant]
...
\stoppagecolumns        
\stoptyping
\index{startpagecolumns}

die bessere Wahl.
Dieser Befehl erzeugt bei jedem Wechsel der Spaltendefinition eine neue Seite.
Um einen brauchbaren Zeilenumbruch zu erhalten muss die \type{tolerance} erhöht werden.
Neben \type{tolerant} gibt es noch die Option \type{verytolerant} und \type{stretch}.

\startpagecolumns[zwei]
\setuptolerance[tolerant]
\input knuth
\input knuth
\input knuth
\input knuth
\stoppagecolumns

\stopsection


