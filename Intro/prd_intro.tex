% !TEX TS-program = ConTeXt (LMTX2022)
\startproduct [*]
\project project_intro

% \enablemode[Vim]
% \enablemode[Fonts]

\setupinteraction
  [title={Eine kurze Einführung in ConTeXt},
%   subtitle=SUBTITLE,
   author={Axel Kielhorn},
   keyword={ConTeXt, Einführung}]

%% For PDF/A
\setupbackend[
format={pdf/a-1b:2005}, % or pdf/a-1a:2005
profile={default_cmyk.icc,default_rgb.icc,default_gray.icc},
intent=ISO coated v2 300\letterpercent\space (ECI)]

%% Tagged PDF
%% method=auto ==> default tags by Adobe
\setupbackend[export=yes]
\setupstructure[state=start,method=auto]

\startdocument
  [ title={Eine kurze Einführung in ConTeXt},
   author={Axel Kielhorn},
%    date={Heute},
   ]
\placecontent[criterium=all]

\component [c01_Vorwort]
\component [c03_Installation]
\component [c05_Erstes_Dokument]
\component [c07_Gliederung]
\component [c09_Auflistung]
\component [c11_Aufzählung]
\component [c13_Beschreibungslisten]
\component [c15_Textformatierungen]
\component [c16_Hervorhebungen]
\component [c17_Textgrößen]
\component [c19_Typing]
\component [c21_Textausrichtung]
\component [c23_Verweise]
\component [c25_Links]
\component [c27_Tabellen]
\component [c29_Mathematik]
\component [c31_Einheiten]
\component [c33_Chemie]
\component [c35_Fußnoten]
\component [c37_Bedingte_Übersetzung]
%\component [c38_Mehrspaltensatz]
\component [c39_Grafiken]
\component [c41_Zitate]
\component [c43_Literatur]
\component [c45_Trennungen]
\component [c47_Spielereien]
\component [c49_Ausblick]
\component [c50_Probleme]
\component [c51_Verzeichnisse]


\stopdocument

\stopproduct
