% !TEX TS-program = ConTeXt (LuaTeX 1.0.9)
% !TEX encoding = UTF-8 Unicode

% Copyright 2018 Axel Kielhorn
% Lizenz: CC-BY-SA 4.0 Unported http://creativecommons.org/licenses/by-sa/4.0/deed.de

\startcomponent *
\product prd_intro
\project project_intro

\startsection[
  reference=verzeichnisse,
  title={Verzeichnisse},
  ]

\placecontent[criterium=local]

\startsubsection[title={Inhaltsverzeichnis}]

Das Inhaltsverzeichnis wird entweder mit \type{\completecontent[criterium=all]}
oder mit \type{\placecontent[criterium=all]} erstellt. 
Im ersten Fall gibt es ein Kapitel ohne Nummer, 
das auch nicht im Inhaltsverzeichnis erscheint, im zweiten nicht.

Die Option \type{[criterium]} gibt an, welchen Umfang das Verzeichnis haben soll.
Der Standard ist \type{[criterium=all]} und erzeugt ein Inhaltsverzeichnis für das gesamte Dokument.
Das funktioniert aber nur, wenn der Befehl am Anfang des Dokumentes steht,
es ist daher eine gute Idee die Option \type{[criterium]} immer explizit zu setzen.
Mit dem Wert \type{[local]} wird ein Verzeichnis für die aktuelle Hierarchiestufe, in diesem Fall \type{section}, erstellt. 

\useURL[contextcontent][https://wiki.contextgarden.net/Table_of_Contents][][Contextgarden]

Es gibt viele Möglichkeiten das Aussehen des Inhaltsverzeichnises zu beeinflussen.
Im \from[contextcontent] gibt es dazu eine umfangreiche Seite.

\usesymbols[mvs]
\setupsymbolset[martinvogel 2]

\startsubsubsection[%
title={I \symbol[martinvogel 2][Heart] BX},
list={I like the Bronx},
reference=bronx,
]

Normalerweise erscheint der \type{title} der Gliederungsüberschrift im Inhaltsverzeichnis.
Manchmal ist das nicht sinnvoll, weil der Titel zu lang ist, oder Formeln und andere Sonderzeichen enthält.

Mit der Option \type{list} kann ein alternativer Text für den Verzeichniseintrag gewählt werden.
\starttyping
\startsubsubsection[
  title={I \symbol[martinvogel 2][Heart] BX},
  list={I like the Bronx},
  reference=bronx,
  ]
\stoptyping

Mit der Option \type{marking} kann außerdem der Text bestimmt werden, der in Kopf- oder Fußzeilen erscheint,
wenn man ein Layout benutzt, das die jeweilige Gliederungsüberschrift ausgibt.

\stopsubsubsection

\stopsubsection

\startsubsection[title={Abbildungsverzeichnis}]

Im  \in{Kapitel}[Grafiken] ab \at{Seite}[Grafiken] wurden zwei Abbildungen
dargestellt, die hier im Abbildungsverzeichnis auftauchen sollten.

\placelistoffigures[criterium=all]

\stopsubsection

\startsubsection[title={Tabellenverzeichnis}]

In  \in{Kapitel}[Tabellen] ab \at{Seite}[Tabellen] wurden zwei Tabellen
dargestellt.

\placelistoftables[criterium=all]

\stopsubsection

\startsubsection[title={Schlagwortverzeichnis}]

Es gibt auch ein  Stichwortverzeichnis:
\blank[big]

\placeindex

\stopsubsection

\startsubsection[title={Eigene Verzeichnisse}]

Natürlich kann man auch eigene Verzeichnisse definieren.
Die Definition zeigt nur einige der Optionen.

\starttyping
% Register for date
\defineregister [myregister]
\setupregister  [myregister] [
  indicator=no          %% Sortierbuchstaben ausgeben
      style=sansbold,   %% Überschrift
  textstyle=slanted,    %% Eintrag
  pagestyle=bolditalic, %% Format Seitenzahl
          n=1,          %% Anzahl Spalten
]
\stoptyping

% Register for date
\defineregister [myregister]
\setupregister  [myregister] [
  indicator=no          %% no letter
      style=sansbold,   %% headings
  textstyle=slanted,    %% entries
  pagestyle=bolditalic, %% page refs
          n=1,          %% columns
]

Jetzt steht der Befehl \type{\myregister}\myregister{Myregister} zur Verfügung.
Das Verzeichnis wird mit \type{\placemyregister} ausgegebn.

\placemyregister

\stopsubsection

\startsubsection[title={Literaturverzeichnis}]

Und zuletzt „natürlich“ das Literaturverzeichnis:
\placelistofpublications[intro][method=dataset,numbering=no]

\stopsubsection

\stopcomponent
