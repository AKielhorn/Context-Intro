% !TEX TS-program = ConTeXt (LuaTeX 1.0.9)
% !TEX encoding = UTF-8 Unicode

% Copyright 2018 Axel Kielhorn
% Lizenz: CC-BY-SA 4.0 Unported http://creativecommons.org/licenses/by-sa/4.0/deed.de

\startcomponent *
\product prd_intro
\project project_intro

\startsection[
  reference=verzeichnisse,
  title={Verzeichnisse},
  ]

\startsubsection[title={Abbildungsverzeichnis}]

Im  \in{Kapitel}[Grafiken] ab \at{Seite}[Grafiken] wurden zwei Abbildungen
dargestellt, die hier im Abbildungsverzeichnis auftauchen sollten.

\placelistoffigures[criterium=all]

\stopsubsection

\startsubsection[title={Tabellenverzeichnis}]

In  \in{Kapitel}[Tabellen] ab \at{Seite}[Tabellen] wurden zwei Tabellen
dargestellt.

\placelistoftables[criterium=all]

\stopsubsection

\startsubsection[title={Schlagwortverzeichnis}]

Es gibt auch ein  Stichwortverzeichnis:
\blank[big]

\placeindex

\stopsubsection

\startsubsection[title={Eigene Verzeichnisse}]

Natürlich kann man auch eigene Verzeichnisse definieren.
Die Definition zeigt nur einige der Optionen.

\starttyping
% Register for date
\defineregister [myregister]
\setupregister  [myregister] [
  indicator=no          %% Sortierbuchstaben ausgeben
      style=sansbold,   %% Überschrift
  textstyle=slanted,    %% Eintrag
  pagestyle=bolditalic, %% Format Seitenzahl
          n=1,          %% Anzahl Spalten
]
\stoptyping

% Register for date
\defineregister [myregister]
\setupregister  [myregister] [
  indicator=no          %% no letter
      style=sansbold,   %% headings
  textstyle=slanted,    %% entries
  pagestyle=bolditalic, %% page refs
          n=1,          %% columns
]

Jetzt steht der Befehl \type{\myregister}\myregister{Myregister} zur Verfügung.
Das Verzeichnis wird mit \type{\placemyregister} ausgegebn.

\placemyregister

\stopsubsection

\startsubsection[title={Literaturverzeichnis}]

Und zuletzt „natürlich“ das Literaturverzeichnis:
\placelistofpublications[intro][method=dataset,numbering=no]

\stopsubsection

\stopcomponent
