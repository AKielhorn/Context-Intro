% !TEX TS-program = ConTeXt (LuaTeX 1.0.9)
% !TEX encoding = UTF-8 Unicode

% Copyright 2018 Axel Kielhorn
% Lizenz: CC-BY-SA 4.0 Unported http://creativecommons.org/licenses/by-sa/4.0/deed.de

\startcomponent *
\product prd_intro
\project project_intro

\startsection[
  reference=mathematik,
  title={Mathematik},
  ]
\index{Mathematik} \index{Formeln}
ConTeXt wird häufig im naturwissenschaftlichen Umfeld benutzt, in der viel
mit mathematischen Formeln hantiert wird.
In den meisten Fällen verhält sich ConTeXt wie LaTeX,
daher kann die reichlich vorhandene Literatur hier mitbenutzt werden.

Formeln im Fließtext erreicht man mittels 
\starttyping
$…$
\m{…}
\math{…}
\mathematics{…}
\stoptyping

Abgesetzte Formeln mittels \type{\startformula…\stopformula}.

Das Paket AMSTeX bietet in LaTeX viele Möglichkeiten Formel zu gestalten.
Im Dokument 
\goto{\hyphenatedurl{http://dl.contextgarden.net/myway/context-latex-math.pdf}}
   [url(http://dl.contextgarden.net/myway/context-latex-math.pdf)]
werden die LaTeX und ConTeXt Befehle gegenübergestellt.

\startsubsection[title={Ausrichtung}]

Einzelne Formeln stehen in einer Zeile.
Der Befehl \type{\placeformula} erzeugt dabei die Nummerierung.
Gibt man dem Befehl einen Formelnamen als Argument, 
kann sich im Text darauf beziehen, siehe Formel \in[formel1].

\starttyping
\placeformula[formel1]
\startformula
  v = u + at
\stopformula
\stoptyping

\placeformula[formel1]
\startformula
v = u + at
\stopformula

Die gleiche Formel ohne \type{\placeformula}:

\startformula
v = u + at
\stopformula

Mehrere Formeln werden zentriert untereinander geschrieben.
Die einzelnen Formeln werden dabei von \type{\NC … \NR} umschlossen. 

\starttyping
\placeformula \startformula
  \startmathalignment[n=1]
    \NC v = u + at,           \NR
    \NC d = ut + \frac{1}{2} at^2 \NR
  \stopmathalignment
\stopformula
\stoptyping

In diesem Fall wird für die Gruppe eine Nummer vergeben.

\placeformula \startformula
  \startmathalignment[n=1]
    \NC v = u + at,           \NR
    \NC d = ut + \frac{1}{2} at^2 \NR
  \stopmathalignment
\stopformula

Um alle (oder nur einzelne) Formel zu nummerieren, 
muss die Zeile, die eine Nummer erhalten soll, mit \type{\NR[+]} abgeschlossen werden.

\placeformula \startformula
  \startmathalignment[n=1]
    \NC v = u + at,           \NR[+]
    \NC d = ut + \frac{1}{2} at^2 \NR
  \stopmathalignment
\stopformula

Mit \type{\startmathalignment[number=auto]} werden die Formeln einzeln nummeriert.

\placeformula \startformula
  \startmathalignment[n=1,number=auto]
    \NC v = u + at,           \NR
    \NC d = ut + \frac{1}{2} at^2 \NR
  \stopmathalignment
\stopformula

Die Formeln lassen sich auch linksbündig \type{[align=left]} oder rechtsbündig \type{[align=right]} ausrichten.
Wichtiger ist aber die Ausrichtung nach Spalten.

\starttyping
\startmathalignment
  \NC v \NC = u + at,           \NR
  \NC d \NC = ut + \frac{1}{2} at^2 \NR
\stopmathalignment

\stoptyping

\startformula
  \startmathalignment
\NC v \NC = u + at,           \NR
\NC d \NC = ut + \frac{1}{2} at^2 \NR
  \stopmathalignment
\stopformula

Es können auch mehrere Spalten definiert werden, 
in diesem Fall muss die Anzahl der Spalten angegeben werden.

\starttyping
\startmathalignment[n=3]
  \NC A + B \NC = C + D  \NC + F     \NR
  \NC G     \NC = C      \NC + F + G \NR
\stopmathalignment
\stoptyping

\startformula
  \startmathalignment[n=3]
    \NC A + B \NC = C + D  \NC + F     \NR
    \NC G     \NC = C      \NC + F + G \NR
  \stopmathalignment
\stopformula

\stopsubsection
\startsubsection[title={Exponenten und Indizes}]

Exponenten gelten nur im Mathematikmodus.
Sie werden durch ein \type{^} und Indizes durch ein \type{_} gekennzeichnet
und lassen sich mit \type{{}} gruppieren.

Einfache $a^b_c$, mehrfache  $a^{ab}_{ac}$ und mehrstufige  $a^{b^c}_d$ Exponenten.

Als Index wird oft ein Wort anstelle eines Formelzeichens verwendet.
Dies muss als \type{\text{}} eingegeben werden.

\type{$P_{\text{Antrieb}}$} $P_{\text{Antrieb}}$ statt \type{$P_{Antrieb}$} $P_{Antrieb}$.

\stopsubsection
\startsubsection[title={Brüche und Klammern}]

Der \type{\frac{}{}} Befehl setzt einen normalen Bruch, 
der Befehl \type{\binom{}{}} schreibt beide Argumente ohne Bruchstrich und fügt eine Klammer in der richtigen Größe hinzu.

\starttyping
\frac{ab}{cd} \binom{ab}{cd}
\stoptyping

\startformula
\frac{ab}{cd} \binom{ab}{cd}
\stopformula

Klammern lassen sich auch aus einem Paar \type{\left( \right)} Befehlen aufbauen.
Die Befehle müssen immer zusammen benutzt werden, ist kein zweites Symbol gewünscht, kann ein \type{.} benutzt werden.

\starttyping
\left( \frac{ab}{cd} \right) \left(  ab \right)\NR
\left{ \frac{ab}{cd} \right.
\stoptyping

\startformula
  \left( \frac{ab}{cd} \right) \left(  ab \right)\NR
  \left{ \frac{ab}{cd} \right.
\stopformula

\stopsubsection

\startsubsection[title={Griechische Buchstaben}]
\index{Griechische Buchstaben}

Die griechischen Buchstaben funktionieren auch im Textmodus.
Es gibt für alle Großbuchstaben einen Befehl, auch für die, bei denen der
griechische Buchstabe genauso aussieht, wie der lateinische.

\blank[big]
{
\bTABLE
\setupTABLE[frame=off]
\setupTABLE[c][width=1.6cm]
\bTR \bTD         \eTD \bTD [nc=2] Textmodus \eTD \bTD [nc=2] Mathemodus \eTD \bTD Varianten \eTD \eTR
\bTR \bTD Alpha   \eTD \bTD \Alpha           \eTD \bTD \alpha            \eTD \bTD $\Alpha$   \eTD \bTD $\alpha$   \eTD \eTR
\bTR \bTD Beta    \eTD \bTD \Beta            \eTD \bTD \beta             \eTD \bTD $\Beta$    \eTD \bTD $\beta$    \eTD \eTR
\bTR \bTD Gamma   \eTD \bTD \Gamma           \eTD \bTD \gamma            \eTD \bTD $\Gamma$   \eTD \bTD $\gamma$   \eTD \eTR
\bTR \bTD Delta   \eTD \bTD \Delta           \eTD \bTD \delta            \eTD \bTD $\Delta$   \eTD \bTD $\Delta$   \eTD \eTR
\bTR \bTD Epsilon \eTD \bTD \Epsilon         \eTD \bTD \epsilon          \eTD \bTD $\Epsilon$ \eTD \bTD $\epsilon$ \eTD \bTD \varepsilon \eTD\eTR
\bTR \bTD Zeta    \eTD \bTD \Zeta            \eTD \bTD \zeta             \eTD \bTD $\Zeta$    \eTD \bTD $\zeta$    \eTD \eTR
\bTR \bTD Eta     \eTD \bTD \Eta             \eTD \bTD \eta              \eTD \bTD $\Eta$     \eTD \bTD $\eta$     \eTD \eTR
\bTR \bTD Theta   \eTD \bTD \Theta           \eTD \bTD \theta            \eTD \bTD $\Theta$   \eTD \bTD $\theta$   \eTD \bTD \vartheta \eTD\eTR
\bTR \bTD Iota    \eTD \bTD \Iota            \eTD \bTD \iota             \eTD \bTD $\Iota$    \eTD \bTD $\iota$    \eTD \eTR
\bTR \bTD Kappa   \eTD \bTD \Kappa           \eTD \bTD \kappa            \eTD \bTD $\Kappa$   \eTD \bTD $\kappa$   \eTD \eTR
\bTR \bTD Lambda  \eTD \bTD \Lambda          \eTD \bTD \lambda           \eTD \bTD $\Lambda$  \eTD \bTD $\lambda$  \eTD \eTR
\bTR \bTD Mu      \eTD \bTD \Mu              \eTD \bTD \mu               \eTD \bTD $\Mu$      \eTD \bTD $\mu$      \eTD \eTR
\bTR \bTD Nu      \eTD \bTD \Nu              \eTD \bTD \nu               \eTD \bTD $\Nu$      \eTD \bTD $\nu$      \eTD \eTR
\bTR \bTD Xi      \eTD \bTD \Xi              \eTD \bTD \xi               \eTD \bTD $\Xi$      \eTD \bTD $\xi$      \eTD \eTR
\bTR \bTD Omicron \eTD \bTD \Omicron         \eTD \bTD \omicron          \eTD \bTD $\Omicron$ \eTD \bTD $\omicron$ \eTD \eTR
\bTR \bTD Pi      \eTD \bTD \Pi              \eTD \bTD \pi               \eTD \bTD $\Pi$      \eTD \bTD $\pi$      \eTD \bTD \varpi \eTD\eTR
\bTR \bTD Rho     \eTD \bTD \Rho             \eTD \bTD \rho              \eTD \bTD $\Rho$     \eTD \bTD $\rho$     \eTD \bTD \varrho \eTD\eTR
\bTR \bTD Sigma   \eTD \bTD \Sigma           \eTD \bTD \sigma            \eTD \bTD $\Sigma$   \eTD \bTD $\sigma$   \eTD \bTD \varsigma \eTD\eTR
\bTR \bTD Tau     \eTD \bTD \Tau             \eTD \bTD \tau              \eTD \bTD $\Tau$     \eTD \bTD $\tau$     \eTD \eTR
\bTR \bTD Upsilon \eTD \bTD \Upsilon         \eTD \bTD \upsilon          \eTD \bTD $\Upsilon$ \eTD \bTD $\upsilon$ \eTD \eTR
\bTR \bTD Phi     \eTD \bTD \Phi             \eTD \bTD \phi              \eTD \bTD $\Phi$     \eTD \bTD $\phi$     \eTD \eTR
\bTR \bTD Chi     \eTD \bTD \Chi             \eTD \bTD \chi              \eTD \bTD $\Chi$     \eTD \bTD $\chi$     \eTD \eTR
\bTR \bTD Psi     \eTD \bTD \Psi             \eTD \bTD \psi              \eTD \bTD $\Psi$     \eTD \bTD $\psi$     \eTD \eTR
\bTR \bTD Omega   \eTD \bTD \Omega           \eTD \bTD \omega            \eTD \bTD $\Omega$   \eTD \bTD $\omega$   \eTD \eTR
\eTABLE
}

\stopsubsection

\startsubsection[title={Beispielformeln}]


Dies ist eine komplizierte Formel:
\startformula 
    t-t_{0}=\sqrt{\frac{l}{g}}\int_{0}^{\varphi}{\frac{d\psi}{\sqrt{1-k^{2}\sin^{2} {\psi}}}} = \left(\sqrt{\frac{l}{g}} F(k,\varphi)\right) 
\stopformula

\stopsubsection
\stopsection


\stopcomponent

