% !TEX TS-program = ConTeXt (LMTX2022)
% !TEX encoding = UTF-8 Unicode

% Copyright 2018 Axel Kielhorn
% Lizenz: CC-BY-SA 4.0 Unported http://creativecommons.org/licenses/by-sa/4.0/deed.de

\startcomponent *
\product prd_intro
\project project_intro

\startsection[
  reference=textgrößen,
   title={Textgrößen},
  ]

Als nächstes sollen verschiedene Textgrößen ausprobiert werden.

\startitemize
\item
  \type{\tfxx} {\tfxx Zwei Stufen verkleinert}
\item
  \type{\tfx} {\tfx Eine Stufe verkleinert}
\item
  \type{\tf} normale Größe
\item
  \type{\tfa} {\tfa Eine Stufe vergrößert}
\item
  \type{\tfb} {\tfb Zwei Stufe vergrößert}
\item
  \type{\tfc} {\tfc Drei Stufe vergrößert}
\item
  \type{\tfd} {\tfd Vier Stufe vergrößert}
\item
  \type{\bfa} {\bfa Eine Stufe vergrößert und fett}
\item
  \type{\ita} {\ita Eine Stufe vergrößert und kursiv}
\item
  \type{\ssa} {\ssa Eine Stufe vergrößert und Sans Serif}
\stopitemize

Schriftgrößen gelten alle blockweise, also bitte {\em nicht} so etwas
wie \type{\tfa{Gross und klein}} schreiben, weil
das Ergebnis so etwas ist: {\tfa Gross und klein.}

\stopsection

\stopcomponent

