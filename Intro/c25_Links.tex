% !TEX TS-program = ConTeXt (LMTX2022)
% !TEX encoding = UTF-8 Unicode

% Copyright 2018 Axel Kielhorn
% Lizenz: CC-BY-SA 4.0 Unported http://creativecommons.org/licenses/by-sa/4.0/deed.de

\startcomponent *
\product prd_intro
\project project_intro

\startsection[
  reference=links,
  title={Links},
  ]
\index{Links}
Für Hyperlinks ist kein zusätzliches Paket erforderlich, das Dokument muss lediglich mit

\starttyping
\setupinteraction[state=start]
\stoptyping

als interaktiv definiert werden.

Normalerweise definiert man die URLs direkt bei der Benutzung.
Der Text in der geschweiften Klammer wird ausgegeben, 
das Argument in der eckigen Klammer ist der eigentliche Link.

\starttyping
\goto{\hyphenatedurl{https://wiki.contextgarden.net/}}
   [url(https://wiki.contextgarden.net/)]
\goto{ConTeXt garden}[url(https://wiki.contextgarden.net/)]
\stoptyping

\goto{\hyphenatedurl{https://wiki.contextgarden.net/}}
   [url(https://wiki.contextgarden.net/)]
   
\goto{ConTeXt garden}[url(https://wiki.contextgarden.net/)]

Alternativ kann man die URLs mit dem \tex{useURL} Befehl definieren und dann mit dem \tex{from} Befehl nutzen.
Das ist dann sinnvoll, wenn eine URL mehrfach benötigt.
Außerdem lassen sich so alle URLs an einer Stelle (z.\,B in einer Environmentdatei) speichern.
Somit hat man nur eine Stelle die aktualisiert werden muss.

Will man eine URL in einer Bildunterschrift (z.\,B. im \tex{placefigure} Befehl) nutzen, dann muss diese Form benutzt werden.
(Siehe \in{Kapitel}[Grafiken])

\starttyping
\useURL[url1][http://www.dante.de/]
\useURL[url2][http://www.dante.de/][][DANTE e.V.]
\from[url1]
\from[url2]
\stoptyping

\useURL[url1][http://www.dante.de/]
\useURL[url2][http://www.dante.de/][][DANTE e.V.]

\from[url1]

\from[url2]

\stopsection

\stopcomponent

