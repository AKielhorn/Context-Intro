% !TEX TS-program = ConTeXt (LuaTeX 1.0.9)
% !TEX encoding = UTF-8 Unicode

\startcomponent *
\product prd_intro
\project project_intro

\startsection[
  reference=typing,
  title={Typing},
  ]
\index{Verbatim}\index{Typing}

\type{Typing} soll den Text exakt wiedergeben. Das heißt, mehrere
Leerzeichen werden nicht zu einem Leerraum zusammengefügt, ConTeXt-Befehle
werden nicht ersetzen und Kommentare nicht ausgeblendet.

Für kurze Einschübe in der Zeile dient der Befehl \type{\type{}}.

In LaTeX heißt die Umgebung \type{verbatim}.

\starttyping
% Ein Kommentar 
\startitemize
    \item {\tiny tiny}
\stopitemize
\stoptyping

\startsubsection[
  reference=syntax,
  title={Syntaxhervorhebung}]

ConTeXt kann verschiedene Syntaxelemente farblich hervorheben.
Dazu muss eine neuer \type{\typing}-Befehl definiert werden,
damit die richtige Sprache ausgewählt wird.

Unterstützt werden \type{C}, \type{LUA}, \type{MP} (MetaPost und MetaFont), \type{TEX} und \type{XML}.

Das Modul \type{scite} unterstützt zusätzlich noch \type{btx} (BibTeX), \type{sql} und \type{JSON}.

Die größte Auswahl an Syntaxdefinitionen bietet das Modul \type{vim}.
Um es zu benutzen muss ein Vim mit den entsprechenden Definitionen installiert sein.
Unter Linux und MacOS (bis Version 10.14) ist das normalerweise der Fall.

In diesem Beispiel wird MetaPost (abgekürzt \type{MP} bzw. \type{mp}) benutzt:

Dazu wird der Farbmodus eingeschaltet und die sprachspezifischen Befehle definiert:

\starttyping
  \setupcolors[state=start]
  \definetyping [MP] [option=MP]
  \usemodule[vim]
  \definevimtyping[MPV][syntax=mp] % MP Vim
\stoptyping

\setupcolors[state=start]
\definetyping [MP] [option=MP]
%\usemodule[vim]
\definevimtyping[MPV][syntax=mp]


Zuerst nur die ConText Standard-Version:

\starttyping
  \startMP
  draw (0,0) withpen pencircle 1in withcolor red ;
  \stopMP
\stoptyping

\startMP
draw (0,0) withpen pencircle 1in withcolor red ;
\stopMP

Dann mit Scite (Der \type{\usemodule}-Befehl gehört normalerweise an den Anfang des Dokuments, bzw. in die Environment-Datei.):

\starttyping
  \usemodule[scite]
  \startMP
  draw (0,0) withpen pencircle 1in withcolor red ;
  \stopMP
\stoptyping

\usemodule[scite]

\startMP
draw (0,0) withpen pencircle 1in withcolor red ;
\stopMP

Und zum Abschluss mit Vim:

\starttyping
  \startMPV
  draw (0,0) withpen pencircle 1in withcolor red ;
  \stopMPV
\stoptyping

\startMPV
draw (0,0) withpen pencircle 1in withcolor red ;
\stopMPV

\stopsection

\stopcomponent
