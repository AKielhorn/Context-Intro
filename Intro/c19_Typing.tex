% !TEX TS-program = ConTeXt (LuaTeX 1.0.9)
% !TEX encoding = UTF-8 Unicode

\startcomponent *
\product prd_intro
\project project_intro

\startsection[
  reference=typing,
  title={Typing}
  ]
\index{Verbatim}\index{Typing}

\type{Typing} soll den Text exakt wiedergeben. Das heißt, mehrere
Leerzeichen werden nicht zu einem Leerraum zusammengefügt, ConTeXt-Befehle
werden nicht ersetzen und Kommentare nicht ausgeblendet.

Für kurze Einschübe in der Zeile dient der Befehl \type{\type{}}.

In LaTeX heißt die Umgebung \type{verbatim}.

\starttyping
% Ein Kommentar 
\startitemize
    \item {\tiny tiny}
\stopitemize
\stoptyping

\stopsection

\stopcomponent
