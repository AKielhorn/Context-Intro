% !TEX TS-program = ConTeXt (LMTX2022)
% !TEX encoding = UTF-8 Unicode

% Copyright 2018 Axel Kielhorn
% Lizenz: CC-BY-SA 4.0 Unported http://creativecommons.org/licenses/by-sa/4.0/deed.de

\startcomponent *
\product prd_intro
\project project_intro

\startsection[
  reference=Ausblick,
  title={Ausblick},
  ]

\startsubsection[title={Registerhaltiger Satz}]
\index{registerhaltiger Satz}\index{Grundlinienraster}
ConTeXt bietet Unterstützung für registerhaltigen Satz.
Dadurch werden die Grundlinien der Zeilen an einem Raster ausgerichtet.
Das ist besonders bei mehrspaltigem Satz sinnvoll.

Der registerhaltige Satz wird mit

\starttyping
\setuplayout[grid=yes]
\stoptyping

aktiviert.
Damit das Ergebnis gut aussieht, müssen an verschiedenen Stellen Anpassungen im Layout vorgenommen werden.
So sollten Überschriften ein Vielfaches des Grundlinienraster hoch sein, 
auch wenn sie auf mehrere Zeile umbrochen werden.

\stopsubsection

\startsubsection[title={Titelblatt}]

\index{Titelblatt}

Diese Dokument benutzt das \type{wtitle} Modul um ein Titelblatt zu erzeugen.
Da kein Datum definiert wurde, wird das Datum der PDF-Erstellung benutzt.

\starttyping
\doctitle{Eine kurze Einführung in ConTeXt}
\author{Axel Kielhorn}
%\date{}
\maketitle
\stoptyping

\stopsubsection


%\startparagraph
%
%\stopparagraph

\startsubsection[
  reference=Abkürzung,
  title={Abkürzung},
  ]

Das ist alles so kompliziert, geht es nicht auch einfacher?

Natürlich würde ich diese Frage nicht stellen, wenn es keine Antwort darauf gäbe.

\startsubsection[
  reference=Pandoc,
  title={Pandoc},
  ]

\index{Pandoc}  
Das Program \goto{Pandoc}[url(https://pandoc.org)] konvertiert (unter anderem) Markdown in viele Formate.

Ein Zielformat ist \ConTeXt\. So kann man den Text zuerst in Markdown schreiben, 
nach \ConTeXt\ konvertieren und dann die Feinheiten im erzeugten Text bearbeiten.

\stopsubsection

\startsubsection[
  reference=Org-Mode,
  title={Org-Mode},
  ]

\index{Org Mode}
Für Anhänger des Org Modes gibt es seit kurzem einen 
\goto{ConTeX Exporter}[url(https://github.com/Jason-S-Ross/ox-context)]
mit dem man direkt \ConTeXt\ exportieren kann.

Vorher ging das nur mit einem Zwischenschritt über Pandoc.

\stopsubsection

\stopsubsection


\stopsection

\stopcomponent

