% !TEX TS-program = ConTeXt (LuaTeX 1.0.9)
% !TEX encoding = UTF-8 Unicode

% Copyright 2018 Axel Kielhorn
% Lizenz: CC-BY-SA 4.0 Unported http://creativecommons.org/licenses/by-sa/4.0/deed.de

\startcomponent *
\product prd_intro
\project project_intro

\startsection[
  reference=Ausblick,
  title={Ausblick}
  ]

\startsubsection[title={Registerhaltiger Satz}]
\index{registerhaltiger Satz}\index{Grundlinienraster}
ConTeXt bietet Unterstützung für registerhaltigen Satz.
Dadurch werden die Grundlinien der Zeilen an einem Raster ausgerichtet.
Das ist besonders bei mehrspaltigem Satz sinnvoll.

Der registerhaltige Satz wird mit

\starttyping
\setuplayout[grid=yes]
\stoptyping

aktiviert.
Damit das Ergebnis gut aussieht, müssen an verschiedenen Stellen Anpassungen im Layout vorgenommen werden.
So sollten Überschriften ein Vielfaches des Grundlinienraster hoch sein, 
auch wenn sie auf mehrere Zeile umbrochen werden.

\stopsubsection

\startsubsection[title={Titelblatt}]

\index{Titelblatt}

Diese Dokument benutzt das \type{wtitle} Modul um ein Titelblatt zu erzeugen.
Da kein Datum definiert wurde, wird das Datum der PDF-Erstellung benutzt.

\starttyping
\doctitle{Eine kurze Einführung in ConTeXt}
\author{Axel Kielhorn}
%\date{}
\maketitle
\stoptyping

\stopsubsection

\startsubsection[title={Inhaltsverzeichnis}]

\index{Inhaltsverzeichnis}

Das Inhaltsverzeichnis wird entweder mit \type{\completecontent}
oder mit \type{\placecontent} erstellt. Im ersten Fall gibt es eine Überschrift, im zweiten nicht.

\useURL[contextcontent][https://wiki.contextgarden.net/Table_of_Contents][][Contextgarden]

Es gibt viele Möglichkeiten das Aussehen des Inhaltsverzeichnises zu beeinflussen.
Im \from[contextcontent] gibt es dazu eine umfangreiche Seite.
 

\stopsubsection


\startsubsection[title={Literaturverzeichnis}]

\index{Literaturverzeichnis}
ConTeXt kann ein Literaturverzeichnis erstellen und kommt dabei ohne externe Programme aus.

\stopsubsection

\startsubsection[title={Schlagwortregister}]

\index{Index}\index{Schlagwortregister}
ConTeXt kann ein Schlagwortregister (Index) erstellen und kommt dabei ohne externe Programme aus.

\stopsubsection

%\startparagraph
%
%\stopparagraph

\stopsection

\stopcomponent

