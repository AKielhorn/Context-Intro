% !TEX TS-program = ConTeXt (LuaTeX 1.0.9)
% !TEX encoding = UTF-8 Unicode

% Copyright 2018 Axel Kielhorn
% Lizenz: CC-BY-SA 4.0 Unported http://creativecommons.org/licenses/by-sa/4.0/deed.de

\startcomponent *
\product prd_intro
\project project_intro

\startsection[
  reference=chemie,
  title={Chemie}
  ]
\index{Chemie}
Zur Eingabe chemischer Formeln benutzt man den \type{\chemical} Befehl.
Leider ist dieser Bereich schlecht dokumentiert.

\starttyping
\startchemicalformula
 \chemical{2 H_2}\chemical{PLUS}\chemical{O_2}
 \chemical{GIVES}\chemical{2 H_2O}
\stopchemicalformula
\stoptyping

\startchemicalformula
 \chemical{2 H_2}\chemical{PLUS}\chemical{O_2}\chemical{GIVES}\chemical{2 H_2O}
\stopchemicalformula

Das kann man auch als Textformel darstellen: \chemical{2 H_2,PLUS,O_2,GIVES,2 H_2O}.

Wasser dissoziiert zu Ionen.

\starttyping
\startchemicalformula
 \chemical{2 H_2O}\chemical{GIVES}
 \chemical{OH\high{$-$}}\chemical{PLUS}\chemical{H_3O\high{$+$}}
\stopchemicalformula
\stoptyping

\startchemicalformula
 \chemical{2 H_2O}\chemical{GIVES}\chemical{OH\high{$-$}}\chemical{PLUS}\chemical{H_3O\high{$+$}}
\stopchemicalformula

Zum Abschluss noch etwas Süßes

\startchemical [height=4500,bottom=2500]
  \bottext{$\beta$-D-Fructofuranose}
  \chemical[FIVEFRONT,BB123,+SB4,-SB5,Z5][O]
  \chemical[+LR1234,+LRZ1234][OH,\SR{HO},H,\SR{HOH_2C}] % oben
  \chemical[+RR1234,-RRZ1234][\SL{CH_2OH},H,OH,H] %unten
\stopchemical

\startchemical [height=4500,bottom=2500]
  \bottext{$\beta$-D-Fructopyranose}
  \chemical[SIXFRONT,BB123,B4,+SB5,-SB6,Z6] [O]
  \chemical[+LR12345,LRZ12345] [OH,\SR{HO},H,H,H] % oben
  \chemical[+RR12345,RRZ12345] [CH_2OH,H,\SL{OH},\SR{HO},H] % unten
\stopchemical


\stopsection

\stopcomponent

