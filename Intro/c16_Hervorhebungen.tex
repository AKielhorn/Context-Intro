% !TEX TS-program = ConTeXt (LuaTeX 1.0.9)
% !TEX encoding = UTF-8 Unicode

% Copyright 2018 Axel Kielhorn
% Lizenz: CC-BY-SA 4.0 Unported http://creativecommons.org/licenses/by-sa/4.0/deed.de

\startcomponent *
\product prd_intro
\project project_intro

\startsection[
  reference=hervorhebungen,
  title={Hervorhebungen},
  ]

Es gibt mehrere Möglichkeiten Textblöcke hervorzuheben.
So kann man sie z.\,B. mit einem breiteren Rand setzen.

Mit dem Befehl \type{\startnarrower} wird der Rand standardmäßig auf beiden Seiten um \unit{5 mm} verbreitert,
der Textblock wird somit \unit{1 cm} schmaler.

\startnarrower
\input knuth
\stopnarrower

Die Breite lässt sich für beide Ränder getrennt einstellen:

\starttyping
\setupnarrower[left=1cm,right=3cm]
\startnarrower[left,right]
\input knuth
\stopnarrower\stoptyping

\setupnarrower[left=1cm,right=3cm]
\startnarrower[left,right]
\input knuth
\stopnarrower

Oder für beide Seiten gleich.
Die Option \type{[middle]} muss beim \type{\startnarrower}-Befehl nicht angegeben werden,
da es die Standardeinstellung ist.\index{narrower}

\starttyping
\setupnarrower[middle=1cm]
\startnarrower
\input knuth
\stopnarrower\stoptyping

\setupnarrower[middle=1cm]
\startnarrower
\input knuth
\stopnarrower

Sehr viel auffälliger ist es jedoch den Text mit einer Markierung im Rand zu versehen:\index{sidebar}

\starttyping
\startsidebar
\input knuth
\stopsidebar
\stoptyping

\startsidebar
\input knuth
\stopsidebar

Natürlich kann man die Farbe und die Position auch verändern:

\starttyping
\startsidebar[rulecolor=red, distance=1cm]
\input knuth
\stopsidebar
\stoptyping

\startsidebar[rulecolor=red, distance=1cm]
\input knuth
\stopsidebar

Aber sowas würde natürlich niemand machen.
Besser ist es zwei neue Stile zu definieren und diese dann zu benutzen:

\starttyping
\definesidebar[neu][rulecolor=green,rulethickness=3pt,distance=6pt]
\definesidebar[alt][rulecolor=red,  rulethickness=3pt,distance=6pt]
\stoptyping

\definesidebar[neu][rulecolor=green, rulethickness=3pt, distance=6pt]
\definesidebar[alt][rulecolor=red,   rulethickness=3pt, distance=6pt]

\startsidebar[alt]
\input knuth
\stopsidebar

\startsidebar[neu]
\input knuth
\stopsidebar

Beide Arten lassen sich auch kombinieren,
dabei muss man darauf achten, die \type{sidebar} innerhalb der \type{narrower} Umgebung zu benutzen.

\startnarrower
\startsidebar
\input knuth
\stopsidebar
\stopnarrower


\stopsection
\stopcomponent