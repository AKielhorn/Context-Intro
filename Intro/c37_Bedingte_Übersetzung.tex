% !TEX TS-program = ConTeXt (LMTX2022)
% !TEX encoding = UTF-8 Unicode

% Copyright 2018 Axel Kielhorn
% Lizenz: CC-BY-SA 4.0 Unported http://creativecommons.org/licenses/by-sa/4.0/deed.de

\startcomponent *
\product prd_intro
\project project_intro


\startsection[
  reference=bedingte-übersetzung,
  title={Bedingte Übersetzung},
  ]

Mit ConTeXt ist es einfach, mehrere Varianten eines Dokuments zu erstellen.
So kann z.\,B. ein Aufgabenbogen für die Studierenden nur die Fragen enthalten,
die Variante für die Korrigierenden enthält zusätzlich noch die Antworten.

Da beides aus einem Dokumente erstellt wird, bleiben Fragen und Antworten synchron.

\starttyping
\enablemode[antwort]

Frage 1:

\startmode[antwort]
Antwort 1:
\stopmode

\stoptyping

Mit \tex{enablemode[]} wir eine Mode aktiviert, mit \tex{disablemode[]} wird er deaktiviert.
Außerdem kann ein Mode beim Aufruf von ConTeXt auf der Kommandozeile aktiviert werden.

\starttyping
context --mode=antwort
\stoptyping

Mehrere Modes werden durch Komata getrennt.
Modes können nicht geschachtelt werden, der \tex{startmode} gilt bis zum nächsten \tex{stopmode}-Befehl.

Eine andere Anwendung sind mehrsprachige Texte, bei denen die Sprachen einzeln ein- oder ausgeschaltet werden.

\starttyping
\enablemode[deutsch]
\enablemode[englisch]

\startmode[deutsch,englisch]
In beiden Sprachen.
Zum Beispiel für Abbildungen.
\stopmode

\startmode[englisch]
This is in english
\stopmode
\stoptyping

\enablemode[englisch]

\startmode[deutsch,englisch]
In beiden Sprachen.
Zum Beispiel für Abbildungen.
\stopmode

\startmode[englisch]
This is in english
\stopmode

Zusätzlich zu den \tex{startmode}/\tex{stopmode}-Befehlen gibt es noch die invertierte Form \tex{startnotmode}/\tex{stopnotmode}.
Der so markierte Bereich wird ausgeführt, wenn der Mode {\em nicht} definiert ist.

Die Argumente der Befehle \tex{startmode} und \tex{startnotmode} verknüpfen die Argumente mit einer \type{oder}-Funktion,
ist eines der Argumente aktiv (bzw. inaktiv), wird er Inhalt des Modes ausgegeben.

Zusätzlich gibt es die Befehle \tex{startallmodes} und \tex{startnotallmodes}, hier müssen {\em alle} Bedingungen erfüllt sein,
es findet eine \type{und}-Verknüpfung statt.

\starttyping
\startmode[mode1, mode2, ...]
  % Bearbeitet wenn ein Mode aktiv ist
\stopmode

\startnotmode[mode1, mode2, ...]
  % Bearbeitet wenn ein Mode inaktiv ist
\stopnotmode

\startallmodes[mode1, mode2, ...]
  % Bearbeitet wenn alle Modes aktiv sind
\stopallmodes

\startnotallmodes[mode1, mode2, ...]
  % Bearbeitet wenn alle Modes inaktiv sind
\stopnotallmodes
\stoptyping

Für kurze Texte gibt es außerdem noch vier \tex{doif}-Befehle

\starttyping
\doifmode        {mode1, mode2} 
                 {Bearbeitet wenn ein Mode aktiv ist}
\doifnotmode     {mode1, mode2} 
                 {Bearbeitet wenn ein Mode inaktiv ist}
\doifallmodes    {mode1, mode2} 
                 {Bearbeitet wenn alle Modes aktiv sind}
\doifnotallmodes {mode1, mode2} 
                 {Bearbeitet wenn alle Modes inaktiv sind}
\stoptyping

sowie drei \tex{doifelse}-Befehle

\starttyping
\doifmodeelse        {mode1, mode2, ...} 
                     {Bearbeitet wenn ein Mode aktiv ist}      
                     {sonst wird dieses bearbeitet} 
\doifallmodeselse    {mode1, mode2, ...} 
                     {Bearbeitet wenn alle Modes aktiv sind}   
                     {sonst wird dieses bearbeitet} 
\doifnotallmodeselse {mode1, mode2, ...} 
                     {Bearbeitet wenn alle Modes inaktiv sind} 
                     {sonst wird dieses bearbeitet}
\stoptyping

\startsubsection[
  title={Version erkennen},
  ]

Neben ConTeXt \MKII\ und \MKIV\ gibt es seit 2019 auch die Version \LMTX.

Manchmal muss man wissen, welche Version gerade benutzt wird,
z.\,B. wenn man Module schreibt, die mit allen Version funktionieren sollen.
Es gibt drei vordefinierte \type{modes} die das überprüfen:

\starttyping
\doifmode        {*mkii}{\MKII}
\doifmode        {*mkiv}{\MKIV}
\doifmode        {*lmtx}{\LMTX}
\stoptyping

Diese Dokument wurde mit
\doifmode        {*mkii}{\MKII\ }%
\doifmode        {*mkiv}{\MKIV\ }%
\doifmode        {*lmtx}{\LMTX\ }%
übersetzt.

\stopsubsection
\stopsection


