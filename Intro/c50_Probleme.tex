% !TEX TS-program = ConTeXt (LMTX2021)
% !TEX encoding = UTF-8 Unicode

% Copyright 2021 Axel Kielhorn
% Lizenz: CC-BY-SA 4.0 Unported http://creativecommons.org/licenses/by-sa/4.0/deed.de

\startcomponent *
\product prd_intro
\project project_intro

\startsection[
  reference=Probleme,
  title={Probleme erschießen},
  ]

In der Mailingliste treten einige Anfragen gehäuft auf.
Daher hier eine kurze Zusammenfassung, damit ich das nicht vergesse.

\startsubsection[title={Zeichensätze}]
\index{Zeichensätze}

Ich kann den Zeichensatz XY nicht mehr benutzen.

Hier hilf es oft, den Font Cache zu aktualisieren:

\starttyping
	mtxrun --script fonts --reload
\stoptyping
oder mit Gewalt:
\starttyping
	mtxrun --script fonts --reload --force
\stoptyping

Eine Liste aller verfügbarer Zeichen erhält man mit:

\starttyping
	mtxrun --script fonts --list --all 
\stoptyping

\starttabulate[|lT|lT|lT|lT|]
\NC identifier		\NC familyname	\NC fontname \NC filename \NR
\NC lmroman10regular \NC latinmodernroman \NC lmroman10regular \NC lmroman10-regular.otf \NR
\stoptabulate

Das hilft oft den genauen Namen herauszufinden unter dem ConTeXt den Zeichensatz kennt.

Am besten leitet man die Ausgabe in eine Date um, da sie doch etwas länger wird.

\stopsubsection

\startsubsection[title={Cache allgemein}]
\index{Cache}

Neben dem Font Cache gibt es noch den allgemeinen Cache für allerlei Sachen.
Manchmal hilft es, den von Hand zu löschen und neu anzulegen:

\starttyping
	mtxrun --script cache --erase && mtxrun --generate
\stoptyping

Wenn es dann immer noch nicht funktioniert, lohnt es sich die Ausgabe anzusehen.
Evtl. zeigt ja ein Pfad noch auf eine veraltete Version.

\stopsubsection


%\startparagraph
%
%\stopparagraph

\stopsection

\stopcomponent

