% !TEX TS-program = ConTeXt (LuaTeX 1.0.9)
% !TEX encoding = UTF-8 Unicode

% Copyright 2018 Axel Kielhorn
% Lizenz: CC-BY-SA 4.0 Unported http://creativecommons.org/licenses/by-sa/4.0/deed.de

\startcomponent *
\product prd_intro
\project project_intro

\startsection[
  reference=spielereien,
  title={Spielereien}
  ]
\index{Initiale}
\startparagraph

\lettrine{D}{ieses} Dokument soll als Beispiel für die Konvertierung von LaTeX nach
EPUB dienen. Wichtig hierbei ist, dass alle Befehle oder Pakete, die
sich auf das Layout beziehen (Satzspiegel, Seitenränder etc.) in der
Regel bei einen EPUB, welches keine feste Seitenstruktur hat, keinen
Sinn ergeben und somit auch von den meisten Konvertierungsprogrammen
ignoriert werden. Aus dem Grund werden auch Veränderungen von Abständen
oder ähnliches nicht geprüft.\cite[stein2000]
\stopparagraph

\startparagraph
\lettrine[FontHook=]{D}{as} Dokument hat neben dieser Einleitung einen Teil mit
Beispielen. In einzelnen Abschnitten werden spezifische Elemente eines
LaTeX-Dokuments beschrieben, die der Konverter dann in das EPUB
übernehmen soll.\cite[mertens2012]
\stopparagraph

\startparagraph
\placeinitial Für das Verständnis der Konvertierung ist es daneben ggf. noch wichtig,
dass ein EPUB zu einem großen Teil einer HTML-Seite mit CSS-Elementen
entspricht. Dementsprechend wird bei der Konvertierung von ConTeXt zuerst
der Zwischenschritt über HTML gegangen. So kann man direkt im Browser
prüfen, ob das Dokument gut aussieht, bevor man sie in ein EPUB
wandelt.\cite[becker2012]
\stopparagraph

\stopsection

\stopcomponent

