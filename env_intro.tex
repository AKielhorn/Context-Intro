\startenvironment env_intro

%\usepath[A]

\usemodule[lettrine]
%\usemodule[typearea]
\usemodule[vim]

%
% Spracheinstellungen
%
\language[de]
\mainlanguage[de]
\setbreakpoints[compound]

\setuphyphenation[method=expanded]

%
% Ligaturen für Aufl* Wörter
%
\replaceword[eka][Auflage][Au{fl}age]
\replaceword[eka][Auflistung][Au{fl}istung]
\setreplacements[eka]

%
% Einstellungen für unit, digits und spaceddigits
%

\let\spaceddigitssymbol,
\setdigitmode {3} \setdigitorder{0} % kleiner 1000er Trenner, Komma
\setupunits[method=3]

%
% Farben
%
\setupcolors[state=start]

%
% Enable hyperlinks
%
\setupinteraction[state=start, color=middleblue]

%
% Papierformat und Layout
%
\setuppapersize [A4][A4]
\setuplayout    [width=middle,  backspace=1.5in, cutspace=1.5in,
                 height=middle, topspace=0.75in, bottomspace=0.75in]


%
% Bilder im ConTeXt Verzeichnis finden.
%
\setupexternalfigures[location=default]

%
% Titelseite
%
\startsetups [document:start]
 %\noheaderandfooterlines
  \blank[force,2*big]
  \startalignment[middle]
    {\definedfont[Regular at 18pt]\documentvariable{title}\par}
    \blank[3*medium]
    {\definedfont[Regular at 14pt]\documentvariable{author}\par}
    \blank[2*medium]
    {\definedfont[Regular at 14pt]\doifelsedocumentvariable{date}{\documentvariable{date}}{\currentdate}\par}
  \stopalignment
  \blank[3*medium]
\stopsetups

%
% Schriften vor typearea
%
\setupbodyfontenvironment[default][em=italic] % default = slanted
%\setupbodyfontenvironment[default][em=bold]
%
\setupalign[hz,hanging]
%
% Fallback für Mathe Symbole
%
\definefallbackfamily[mainface][mm][TeX Gyre Pagella][preset=math:digitsnormal,features=f:oldstyle]
%
% Default: Latin Modern
%
\setupbodyfont[11pt]
%
%\definefontfamily [antykwat] [serif] [AntykwaTorunska] [expansion=quality,protrusion=quality]
%\definefontfamily [antykwat] [sans] [AntykwaTorunska] [expansion=quality,protrusion=quality]
%\setupbodyfont[antykwat]
%
%\definefontfamily [antykwap] [serif] [Antykwa Poltawskiego] [expansion=quality,protrusion=quality]
%\definefontfamily [antykwap] [sans] [Antykwa Poltawskiego] [expansion=quality,protrusion=quality]
%\setupbodyfont[antykwap]
%
%\definefontfamily[latinfamily] [serif][Latin Modern Roman][expansion=quality,protrusion=quality]
%\definefontfamily[latinfamily] [sans][Latin Modern Sans][expansion=quality,protrusion=quality]
%\definefontfamily[latinfamily] [mono][Latin Modern Mono][scale=0.85, features=none]
%\definefontfamily[latinfamily] [math][Latin Modern Math] % für itemize
%\setupbodyfont[latinfamily,11pt]
%
%\definefontfamily[dejafamily] [serif][DejaVu Serif][expansion=quality,protrusion=quality]
%\definefontfamily[dejafamily] [sans][DejaVu Sans][expansion=quality,protrusion=quality]
%\definefontfamily[dejafamily] [mono][Dejavu Sans Mono][scale=0.85, features=none]
%\definefontfamily[dejafamily] [math][DejaVu Math] % für itemize
%\setupbodyfont[dejafamily,11pt]
%
% TeX Gyre Adventor
% TeX Gyre Schola
% TeX Gyre Termes
% TeX Gyre Bonum
% TeX Gyre Chorus
% TeX Gyre Heros
%
%\definefontfamily[pagellafamily] [serif][TeX Gyre Pagella][expansion=quality,protrusion=quality]
%\definefontfamily[pagellafamily] [sans][TeX Gyre Heros][expansion=quality,protrusion=quality]
%\definefontfamily[pagellafamily] [mono][][scale=0.85, features=none]
%\definefontfamily[pagellafamily] [math][TeX Gyre Pagella Math] % für itemize
%\setupbodyfont[pagellafamily,11pt]

% 
% Layout mit typearea 
%
%\setupTypeArea[bcor=3mm, oneside=yes, alphabets=2.6]

\setuppagenumbering[location={footer,center}]
%\setuppagenumbering[alternative=doublesided,location={footer,marginedge}]
%
% header footer
% left middle right margin marginedge inleft inright

%\showframe % Debug Layout

\setupwhitespace[medium]

\setuphead[chapter]             [style=\ssd] % title
\setuphead[section]             [style=\ssc] % subject
\setuphead[subsection]          [style=\ssb] % subsubject
\setuphead[subsubsection]       [style=\bf]  % subsubsubject
\setuphead[subsubsubsection]    [style=\bf]
\setuphead[subsubsubsubsection] [style=\bf]

%\setuphead[chapter, section, subsection, subsubsection][number=no]
\setuphead[subsubsubsection, subsubsubsubsection][number=no]
%\setuphead[subsubsubsection, subsubsubsubsection][number=yes]
\setuphead[subsubsubsection, subsubsubsubsection][alternative=text]

\setupcombinedlist[content][list={chapter,section,subsection,subsubsection}]

%
% Neue Auzählung
%
\defineitemgroup[aufzählung][level=3]
\setupitemgroup[aufzählung][1][n,repeat][stopper=)]
\setupitemgroup[aufzählung][2][a,repeat][stopper=),width=2em]
\setupitemgroup[aufzählung][3][n][width=3em]

%
% Unterschiedliche Descriptions
%
\definedescription
  [description]
  [headstyle=bold, style=normal, location=hanging, width=broad,
   alternative=hanging, distance=0.2em, 
   ]

\definedescription
  [ldescription]
  [headstyle=bold, style=normal, location=hanging, width=broad,
   distance=0.2em, alternative=serried,
   ]

\definedescription
  [xdescription]
  [headstyle=bold, style=normal, location=hanging, width=broad,
   distance=0.2em, alternative=serried,
   ]

\setupitemize[autointro]    % prevent orphan list intro
\setupitemize[indentnext=no]

\setupthinrules[width=15em] % width of horizontal rules

\setupdelimitedtext
  [blockquote]
  [before     = {\blank[medium]},
   after      = {\blank[medium]},
   indentnext = no,
  ]

%\setupdelimitedtext[quotation:1][left=»,right=«]
%\setupdelimitedtext[quotation:2][left=›,right=‹]
%\setupdelimitedtext[quotation:1][left=„,right=“]
%\setupdelimitedtext[quotation:2][left=‚,right=’]

% Quote und Quotation mit "left protrusion" (links hängenden Satzzwichen)
% \setupquote erbt die Einstellung von \setupquotation.
% Wolfgang Schuster E-Mail vom 2020-05-14.

\setupquotation [method=font]
\setupquote     [method=font]

%
% Figure und table 
%
\setupfloat[figure][default={here,nonumber}]
\setupfloat[table][default={here,nonumber}]

%
% Literaturverzeichnis
%
\definebtxdataset[default]
\usebtxdataset[default][epub_latex_beispiel.bib]
\usebtxdataset[default][intro.bib]
\usebtxdefinitions[apa]

%
% Initiale einfach
%
\setupinitial
   [location = text,
    n        = 2,
    color    = darkred,
    distance = 0em,
    hoffset  = 0em,
    voffset  = 3.4ex,
    before   = \blank]

%
% und mit t-lettrine
%
\setuplettrine
   [Lines    = 3, % default 2
    Hang     = .5, % default 0
    Oversize = 0,
    Raise    = 0,
    Findent  = 0pt,
    Nindent  = 0em,
    Slope    = 0em,
    Ante     = ,
    FontHook = \green, % default leer
    TextFont = \kap\bf, % default \sc
    Image    = no]

\stopenvironment
